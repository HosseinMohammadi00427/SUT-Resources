\documentclass{article}
\usepackage{graphicx,xy,amsmath,amssymb,amsthm,physics,mathtools,tcolorbox,hyperref}
\usepackage{xepersian}
\settextfont{XB Niloofar}
\title{	
	پیش گزارش چهارم درس آزمایشگاه اپتیک - دکتر مهدوی
	\\
	\small
	موضوع آزمايش: تداخل به وسیله ی دو شکاف يانگ و دو منشور فرنل
	نوری
}
\author{
حسین محمدی 
\\
۹۶۱۰۱۰۳۵
}
\begin{document}
\maketitle
\section{هدف آزمایش}
	در این آزمایش با دو تداخل سنج بسیار مهم در آزمایشگاه اپتیک آشنا می شویم و شیوه تنظیم کردن و کار با آن ها را می آموزیم؛ به طور خاص این کمیت ها را قرار است مشاهده و اندازه گیری کنیم:
\begin{enumerate}
	\item مشاهده پدیده تداخل در آزمایش یانگ و محاسبه فاصله دو شکاف 
	($a$)
	\item مشاهده پدیده تداخل با دو منشور فرنل و بدست آوردن فاصله دو نوار متوالی
	\item به دست آوردن طول موج لیزر هلیم نئون با کمک روش های اپتیکی (قرار دادن عدسی برای خواندن فاصله دو چشمه مجازی)
	\item اندازه گیری زاویه راس منشور فرنل
\end{enumerate}
\section{معرفی دوشکاف یانگ و دو منشور فرنل}
هنگامی که دو شکاف باریک در مسیر نور قرار بگیرند (عرضشان باید کمتر از طول موج باشد تا پدیده مشاهده شود.) مطابق اصل هویگنس خود این شکاف ها چشمه نور همدوس می شوند که جبهه ی کروی دارد و این امواج می توانند با هم تداخل کنند و  طرح های تداخلی آن ها را می توانیم روی یک پرده به فاصله بسیار دورتر از شکاف مشاهده کنیم. این توصیف بسیار ساده دوشکاف یانگ است.

\noindent\\
دو منشور فرنل از دو منشور با زاویه راس بسیار کوچک تشکیل شده است که از قاعده به هم چسبیده اند و هنگامی که نور به صورت عمود به قاعده بر آن می تابد، بخش پایینی جبهه موج به سمت بالا متمایل می شود و بخش بالایی آن به سمت پایین متمایل می شود و این باعث می شود که بعد از خروج از دو منشور بتوان یک طرح تداخلی را مشاهده کرد، معمولا قبل از این منشور برای تنظیم کردن جبهه موج یک عدسی قرار داده می شود.
\section{تغییر طرح تداخلی در اثر افزایش فاصله دو شکاف}
مطابق رابطه 
$a \sin (\theta) = m \lambda$
،برای مثلا اولین نوار روشن
($m=1$)
 با افزایش 
$a$ ، 
سمت راست عبارت ثابت است پس بایستی مقدار 
$ \sin (\theta)$
کم شود یا به عبارتی بایستی زاویه ای که در آن نوار روشن رویت می شود کم شود، به این معنی که نوار ها فشرده تر می شوند.

\noindent
از طرفی می دانیم که پهنای نوارهای روشن (یا تاریک) از رابطه 
$\frac{\lambda D}{a}$
بدست می آید، پس با افزایش فاصله شکاف ها، پهنای نوار ها کم می شود.
\section{تغییر طرح تداخلی در اثر افزایش عرض دو شکاف}
با افزایش عرض دو شکاف ابتدا پهنای نوارهای تاریک و روشن زیاد می شود و سپس این نوارها محول می شوند،زیرا حدی وجود دارد که از آن حد به بعد رفتار موجی نور  از بین می رود و دقیقا رفتار ذره ای از آن می بینیم، یعنی اگر عرض شکاف از حدی بیشتر شد، دو لکه ی روشن روی پرده دیده می شود و سایر پرده تاریک دیده می شود.


\noindent \\
اما هرچه عرض دو شکاف را کم تر کنیم، پهنای قله ها و دره ها هم کم و کم تر می شود و تفکیک بین دره ها و قله ها واضح تر و بهتر دیده می شود. 













\end{document}