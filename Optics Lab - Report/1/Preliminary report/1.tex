\documentclass{article}
\usepackage{graphicx,xy,amsmath,amssymb,amsthm,physics,mathtools,tcolorbox,hyperref}
\usepackage{xepersian}
\settextfont{XB Niloofar}
\renewcommand{\contentsname}{هفته های تدریس}
\title{	
	پیش گزارش اول درس آزمایشگاه اپتیک - دکتر مهدوی
}
\author{
حسین محمدی 
\\
۹۶۱۰۱۰۳۵
}
\begin{document}
\maketitle
\section{هدف آزمایش}
منشور یکی از پایه ای ترین ابزارها در آزمایشگاه اپتیک است و در این آزمایش سعی می کنیم با این ابزار آشنا شویم و برخی از مشخصه های آن را بدست آوریم.

\noindent\\
به طور خاص تر این کمیت ها را پیدا می کنیم:
\begin{itemize}
	\item ضریب شکست منشور
	\item ضرایب کوشی منشور
\end{itemize}
و در این حین بایستی کمیت های زیر را هم حساب کنیم:
\begin{itemize}
	\item زاویه راس منشور
	$\alpha$
	\item زاویه مینیمم انحراف
	$\delta_m$
\end{itemize}
\section{چگونگی انجام آزمایش}
پیش از هر کاری بایستی طیف سنج را کالیبره یا «تنظیم اولیه» کنیم و برای این کار لامپ را مقابل دوربین موازی ساز قرار می دهیم و دوربین چشمی را طوری تنظیم می کنیم که تصویر واضح و منطبق بر «تار مویی» عمود داشته باشیم و سپس صفر ورنیه را با صفر صفحه مدرج یکی می کنیم.

\noindent\\
در مرحله اول آزمایش ، بایستی راس منشور را اندازه بگیریم: اول  لامپ در مقابل موازی ساز قرار می دهیم و با دوربین چشمی دو باریکه ی بازتاب شده از دو وجه منشور را می بینیم و سپس زاویه را از روی ورنیه می خوانیم. با این دو زاویه می توانیم زاویه راس منشور را بدست آوریم. 

\noindent\\
در مرحله بعدی، با تاباندن نور به منشور و تنظیم کردن دوربین چشمی طیف سنج، و با چرخاندن حامل، سعی می کنیم که زاویه کمینه انحراف را پیدا کنیم و با پیدا کردن این زاویه می توانیم ضریب شکست را بخوانیم و با داشتن ضریب شکست و طول موج نور می توانیم ضرایب کوشی را از روابطی که در جزوه آمده است، بدست بیاوریم.
\section{وسایل مورد نیاز}
به وسایل زیر نیاز داریم تا آزمایش را انجام دهیم:
\begin{itemize}
	\item طیف سنج
	: که به گمان یکی از وسایل پیچیده در این آزمایش بود؛ این وسیله از دو دوربین و چهار پیچ تنظیم برای کالیبره کردن تشکیل شده است و من از روی شکل ها نتوانستم به خوبی با طرز کار آن آشنا شوم. 
	\item منشور شیشه ای
	\item لامپ هلیوم و منبع تغذیه آن
	\item چراغ رومیزی
\end{itemize}
\section{موارد کاربرد}
دانستن ضریب شکست نور برای یک منشور که در آزمایشگاه بسیار کاربرد دارد، اهمیت زیادی دارد. از آنجا که ممکن است در آزمایش های بعدی به این وسیله اپتیکی نیاز داشته باشیم، بهتر است که مشخصه های آن را بدست آوریم.

به علاوه در وسایل اپتیکی مانند تلسکوپ و پریسکوپ از منشور استفاده می شود. وسایل جانبی مختلفی هم هستند که در آن ها نیاز به تجزیه نور است. حتی در بلور لوسترها می توان شکست نور را مشاهده کرد و به یک معنای بلورهای لوستر هم منشور هستند.























\end{document}