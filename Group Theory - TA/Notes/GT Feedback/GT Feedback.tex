\documentclass[]{article}
\usepackage{amsmath,ftnxtra,xcolor,tcolorbox,cancel}
\usepackage{ amssymb, amscd,mdframed, amsthm,amsfonts,bm,fontawesome,xcolor,dsfont}
\usepackage[top=30mm, bottom=25mm, left=1.6in, right=1.6in,includehead=true,includefoot=true]{geometry}
\usepackage[hidelinks,colorlinks=true,allcolors=blue]{hyperref}

\usepackage{xepersian}
\settextfont{XB Niloofar}
\setdigitfont{XB Niloofar}
\parindent 0mm
\newenvironment{parind}{%
	\par%
	\medskip
	\leftskip=0mm\rightskip=7mm
	\noindent\ignorespaces}{%
	\par\medskip}
\newcommand{\comments}[1]{
	
}

\numberwithin{equation}{section}
\newtcolorbox{boxes}[3][]
{
	before upper=\renewcommand\thempfootnote{\fnsymbol{mpfootnote}},
	colframe = #2!25,
	colback  = #2!10,
	coltitle = #2!40!black,  
	title    = {\textbf{#3}},
	#1,
}

%opening
\title{
بازخوردِ دانشجویان در مورد درس نظریه‌گروه‌ها
}

\author{حسین محمدی }
\begin{document}
\maketitle
ما پس از آخرین جلساتِ کلاس، از بچه‌ها خواستیم تا در مورد کلاسِ درس، کلاس‌های حلِ تمرین، استادِ درس، دستیارانِ آموزشی و هرچه‌که به نظرشان می‌رسد‌، به ما بازخوردی بدهند. در این یادداشت نظرات بچه‌ها را دسته‌بندی و تفکیک کرده‌ام. این تحلیل از مجموع ۲۶ نظر تهیه شده؛ اگرچه تعداد نظرات بیشتر بود، اما برخی از این نظرات بیشتر شامل تشکر و تمجید بودند و شاملِ بازخورد یا انتقادی  سازنده در مورد درس نبودند؛ بنابراین من مبنای کارم را برای تهیه‌ی این یادداشت، همان ۲۶ نظر گذاشتم؛ هرچند که این نظرات هم درجای خود ارزشمندند و ما را بسیار خوشحال می‌کند.

\section*{تمجید و تقدیر}
اکثر نظرات شامل تمجید و تشکر بودند، اما آن‌هایی که علتِ تشکرشان به‌خوبی توصیف شده بود، از همه سازنده و دل‌چسب‌تر بود.

از مجموعِ این نظرات، ۱۳ نظر شامل تقدیری از استاد بود. برخوردِ محترمانه، نظمِ ستودنی، دقتِ ریاضی در تدریس و صحبت، کلاسِ لذت‌بخش و پر از شهود و همچنین فنِ بیان عالی از جمله‌ی صفاتی بودند که برای توصیف استاد به‌کار رفتند. سطحِ اثرپذیری دانشجوها از استاد هم در کلامشان مشخص بود. 

 ۱۲ نظر هم به تقدیر از دستیارها اختصاص داشت و صفاتِ نظم، پاسخگویی سریع و کامل، آسان‌گیری و خوش‌رویی برای توصیفِ ما به کار رفت. واقعا خوشحالیم که دریافتِ شما از ما این است.


\section*{نظرات در موردِ درس}
این بخش، از نظرِ من مهم‌ترین رهاوردی است که از بررسی تمامِ این نظرات عایدمان می‌شود. اگرچه بیشترِ نظرات شاملِ انتقادات نسبت به کلاس درسی بود، اما در مورد کلاس‌های حل‌تمرین هم نظرات و پیشنهادهای خوبی گفته شد. برویم سراغِ اصل مطلب.
\subsection*{وضعیتِ کلاسِ درسی}
چند مشکل که در نظرات تکرار شده بود، این‌ها بودند: 
\begin{enumerate}
	\item موشکافیِ بیش‌ازحدِ مفاهیم به‌اندازه‌ای که درس خسته کننده شد. (۴ نفر)
	\item تاکید زیاد روی اثبات قضایا و مثال‌های آموزنده‌ی کم. (۴ نفر)
	\item گنگ و پیچیده شدن مطالب از مقطعی به بعد. (۲ نفر)
\end{enumerate}
البته ناگفته نماند که ۴ نفر هم از وضعیت تدریس و شیوه‌ی کار، کاملا راضی بودند و هیچ‌گونه نظر یا انتقادی نداشتند.
\subsection*{وضعیتِ درس}
این‌جا اما نظرات بیشتر بودند:

\begin{enumerate}
	\item ۱۰ نفر به کند بودنِ پیشروی و همچنین گذرِ کند از مقدمات اولیه اشاره کردند. به نظرم با توجه به تعداد زیاد نظر‌دهندگان درموردِ این مسئله، این یک مسئله‌ی جدی باشد که بایستی برای آینده مرتفع شود.
	\item
	۸ نفر معتقد بودند که پیشروی درس کافی نبود؛ به طور خاص، ۷ نفر اشاره کردند که انتظار داشتند جبرلی و نظریه‌ی نمایش را مفصل‌تر و کامل‌تر بررسی می‌کردیم.
	\item
	یک انتقادِ دیگر که از نظر من وارد است، نپرداختن به ارتباط نظریه‌گروه‌ها با فیزیک است. ۶ نفر هم به این مسئله اشاره کردند. اگرچه من در سطح توانم سعی کردم ارتباط نمایش‌ها با ذرات بنیادی را بشکافم، اما وارد شدن به مباحث عمیق‌تر نظریه میدان مستلزم دانشِ کافی از نظریه نمایش و جبرلی بود.
\end{enumerate}

\subsection*{وضعیتِ کلاسِ حل تمرین}
این‌جا تعداد نظرات کم بود و به‌علت وجود نظرات متناقض، تصمیم‌گیری برای بهبود و اصلاح روند در آینده، آسان نیست.

سه نفر از سختی زیاد تمرینات گفتند و دو نفر از حجم بالای تمرینات انتقاد کردند. اتفاقا دو نفر از تغییر رویه‌مان در وسط ترم انتقاد کردند و رویه‌ی پیشین را مناسب‌تر می‌دانستند. دو نظر هم در مورد سطح خوب تمرینات و آموزنده بودنشان داشتیم. بنابراین این بازخوردها خیلی دقیق وضعیت را مشخص نمی‌کنند؛ هرچند غنیمت هستند!

\section*{نکات متفرقه}
برخی نکات که در هیچ‌یک از عنوان‌های بالا نمی‌گنجند، اینجا آورده شده:
\begin{itemize}
	\item دو نفر اشاره کردند که  استاد پس از اتمام ساعتِ درسی، سریع کلاس را ترک می‌کردند. خوب بود استاد پس از اتمام کلاس اندکی در کلاس می‌ماندند تا بتوانیم سوالات و ابهاماتمان رو بپرسیم.
	
	\item
	یک‌نفر در مورد ناهماهنگی سطح کلاس‌های حل‌تمرین حرف زد که به‌نظرم به‌جا و صحیح است. مشکلات شخصی و کاری ما باعث شد نتوانیم خیلی هماهنگ بشویم؛ اما این بازخورد برای آینده غنیمت است، چون مطالب باید تاحدی هماهنگ باشند که هیچ‌کس استرس شرکت نکردن در کلاسِ دیگر را نداشته باشد.
	\item
	دو نفر اشاره کردند خوب است استراحتی ۵ دقیقه‌ای وسطِ زمان کلاس داشته باشیم؛ انگار سطح انرژی و تمرکزشان در اواسط کلاس پایین می‌آمد.
	
	\item پیشنهاد شد که خوب است سوالاتی که در کلاسِ حلِ تمرین بحث می‌شوند، 
	از قبل در اختیار بچه‌ها قرار بگیرد تا فرصت فکرکردن روی سوالات را داشته باشند؛ رویه کلاس طوری بوده که پس از طرح سوال مستقیما به سراغ حلشان می‌رفتیم.
	\item 
	عده‌ای به من شفاهی گفتند و یک نفر هم نوشته بود که سطح سوالات پایان‌ترم و سختی‌اش اصلا معقول نبود و با گفته‌ی شما که به اندازه‌ی یک
	$\varepsilon$
	  از امتحان میان‌ترم سخت‌تر است، مطابقت نداشت.
	\item 
	دو نفر هم اشاره کردند که دست‌خط استاد با وجود خوانا بودن، ریز است و خوب است که از این به بعد با خطی درشت‌تر روی تخته بنویسید.
	\item 
	دونفر هم اشاره کردند که تغییر رویه وسط ترم(یعنی زیاد شدن زمانِ تاخیر مجاز و کم‌شدن تعداد سوالات و درجه‌سختیِ آن)، خیلی کار را آسان‌تر کرد. همانطور که ما هم بعد از تصحیح نمرات مشاهده کردیم؛ تعداد نمراتِ بیشتر از میانگین بالا بود 
	\footnote{مثلا ۲۴ نفر نمره‌ی حل‌تمرین‌شان بالایِ  ۴.۵  بود؛ سقفِ نمره‌ی تمرین  ۵.۷ بود و این تعداد از نمرات خیلی طبیعی به نظر نمی‌رسد.}
	و برداشت کردیم که انتقادها و اعتراضاتی که آن زمان بود برای کامل شدن نمره‌ی حل‌تمرین بود، نه برای فشار بیش‌ازحد و سخت‌گیری بالا. خلاصه‌ که این تجربه را هم کسب کردم که لازم نیست همواره مطابق نظرات بچه‌ها عمل کنیم؛ اگرچه لازم است حرفشان را بشنویم و با ایشان صحبت کنیم. بنابراین، تنظیم کردن درس و رویه‌ی آن با نظرات بچه‌ها معمولا ایده‌ی خوبی نیست.
\end{itemize}











\end{document}
