\documentclass[a4paper, 12pt]{article}
\usepackage{physics, amsmath, amsfonts, fixmath, geometry, tikz, pgf, multirow, hyperref, amsfonts,amssymb, mathtools, physics, xcolor, siunitx, subcaption,tcolorbox}


\pdfpagewidth 8.5in
\pdfpageheight 11in
\headheight 0pt
\headsep 0pt
\footskip .25in
\marginparwidth 0pt
\marginparsep 0pt
\oddsidemargin \dimexpr 1in -1in
\topmargin \dimexpr 1in -1in
\textwidth \dimexpr \pdfpagewidth -2\oddsidemargin -2in
\textheight \dimexpr \pdfpageheight -2\topmargin -2in
%Commands

\newtcolorbox{boxes}[3][]
{
	colframe = #2!25,
	colback  = #2!10,
	coltitle = #2!40!black,  
	title    = {\textbf{#3}},
	#1,
}
\usepackage{graphicx}
\usepackage{xepersian}
\usepackage{braket}
\settextfont[]{XB Niloofar}
\title{\textbf{
آیا زیرگروه $HK$ آبلی است؟
}}
\author{مربوط به تمرین کلاسی}
\date{}

\begin{document}
\maketitle
در کلاس دیدیم که اگر 
$H$ و $G$ 
دو زیرگروه از گروه $G$ باشند، در این صورت مجموعه‌ی 
$HK$
با تعریف 
\[
HK = \{hk \;\big|\; h\in H , k\in K\}
\]
زیرگروهی از $G$ است، اگر و تنها اگر $HK = KH$.

سوالی که در کلاس مطرح شد این بود که با فرض زیرگروه بودن 
$HK$
‌، آیا زیرگروه آبلی است؟ (چون شرط 
$HK=KH$
ناخودآگاه آبلی بودن را به خاطر می‌آورد.
)

\vspace{1.5em}
\textbf{اولا:}
توجه کنید که معنای شرط 
$HK=KH$
این است که به ازای هر عضو 
$hk \in HK$
حتما اعضایی مثل 
$\tilde{k} \in K , \tilde{h}\in H$
هستند که:
\[
\tilde{k}\tilde{h} = hk
\]
‌‌در حقیقت لزومی ندارد که 
$\tilde{k}$
و 
$\tilde{h}$
همان 
$k$ و $h$ 
باشند. پس انگار شرط ضعیفتری نسبت به آبلی بودن هست.

\vspace{1.5 em}
\textbf{مثال نقض:}
گروه 
$S_3$
را که همگی با آن آشنا هستید، یک مثال نقض ارائه می‌کند. بگیرید:
\begin{equation*}
	\begin{aligned}
		H &= \{e , (12)\} \\
		K &= \{e, (123), (132)\}
	\end{aligned}
\end{equation*}
با تعریف ضرب در گروه $S_3$ می‌توانید ببینید که هر دو زیرگروه هستند و داریم:
\[
HK = \{hk \; \big| \; h\in H , k\in K\} = \{e, (12),(123),(132),(13),(23)\} = S_3
\]
علت اضافه شدن اعضای 
$(13),(23)$
به خاطر ضربهای 
\begin{equation*}
	\begin{aligned}
		(12)(123) &=
		\begin{pmatrix}
			1 & 2 & 3 \\
			2 & 1 & 3
		\end{pmatrix}.
		\begin{pmatrix}
		 	1 & 2 & 3 \\
		 	2 & 3 & 1
		 \end{pmatrix} = 
		 \begin{pmatrix}
		 	1 & 2 & 3 \\
		 	1 & 3 & 2 
		 \end{pmatrix} = 
		 (23)\\
		(12)(132) &=
		\begin{pmatrix}
			1 & 2 & 3 \\
			2 & 1 & 3
		\end{pmatrix}.
		\begin{pmatrix}
			1 & 2 & 3 \\
			3 & 1 & 2
		\end{pmatrix} = 
		\begin{pmatrix}
			1 & 2 & 3 \\
			3 & 2 & 1 
		\end{pmatrix} = 
		 (13)
	\end{aligned}
\end{equation*}
است.
پس می‌بینید که 
$HK = S_3$
آبلی نیست.

\vspace{1.5em}
\noindent
برای آشنایی بیشتر با گروه 
$S_3$  
زیرگروه‌ها، کلاس‌های تزویجی و جدول ضربش، به 
\href{https://groupprops.subwiki.org/wiki/Subgroup_structure_of_symmetric_group:S3}{این}
 و 
 \href{https://groupprops.subwiki.org/wiki/Determination_of_multiplication_table_of_symmetric_group:S3}{این}
  صفحه مراجعه کنید.










\end{document}