\documentclass[a4paper, 12pt]{article}
\usepackage{physics, amsmath, amsfonts, fixmath, geometry, tikz, pgf, multirow, hyperref, amsfonts,amssymb, mathtools, physics, xcolor, siunitx, subcaption,tcolorbox}


\pdfpagewidth 8.5in
\pdfpageheight 11in
\headheight 0pt
\headsep 0pt
\footskip .25in
\marginparwidth 0pt
\marginparsep 0pt
\oddsidemargin \dimexpr 1in -1in
\topmargin \dimexpr 1in -1in
\textwidth \dimexpr \pdfpagewidth -2\oddsidemargin -2in
\textheight \dimexpr \pdfpageheight -2\topmargin -2in
%Commands

\newtcolorbox{boxes}[3][]
{
	colframe = #2!25,
	colback  = #2!10,
	coltitle = #2!40!black,  
	title    = {\textbf{#3}},
	#1,
}
\usepackage{graphicx}
\usepackage{xepersian}
\usepackage{braket}
\settextfont[]{XB Niloofar}
\title{\textbf{
 شیوه‌ی ارزشیابی دانشجویان
}}
\author{درس نظریه گروه‌ها}
\date{}

\begin{document}
\maketitle
سلام.

\noindent

ما تصمیم گرفتیم انتظارات و نحوه‌ی ارزیابی‌تون رو با شما در میون بگذاریم تا با شفافیت بیشتری کارمون رو آغاز کنیم.
\subsection*{تقسیم نمره}
\begin{itemize}
	\item 
	از ۶+۱ نمره‌ای که  در اختیار ماست؛ ۶ نمره مربوط به تمرین‌ها و فعالیت کلاسی هست؛ یعنی به صلاحدید خودمون بخشی از این نمرات رو به فعالیت کلاسی اختصاص میدیم. یک نمره‌هم مربوط به تمرینات امتیازی هست و قراره دو سری تمرین امتیازی به شکل تجمعی داشته باشید.
		\item 
	از اونجایی که نمره‌ی درس از ۲۱ سنجیده میشه؛ این یک نمره‌ی اضافی که دست ماست به نمره‌ی کل شما سرریز میکنه و فرصت خوبیه که کاستی‌های احتمالی امتحاناتتون رو پوشش بده.
	
	\end{itemize}

	
\subsection*{قوانین مربوط به تاخیر }
		\begin{itemize}
	\item 
	هر کسی دوهفته «فرصت تاخیر آزاد» برای تحویل تمرین داره؛ یعنی اگر مجموع تاخیرهاش تا ۱۴ روز باشه، بدون کسر نمره تمرینش تصحیح میشه.
	\item
	اما اگر کسی تاخیرهای مجازش رو خرج کنه و بازم تاخیر داشته باشه، ازش نمره کسر میشه. روز اول نصف نمره کسر میشه و در روز دوم کل نمره از دست میره. 
	
	\item 
	برای درک بهتر، موقعیت های احتمالی زیر رو در نظر بگیرین:
	\begin{itemize}
		\item فرض کنید ۶ سری تمرین داریم و من با تاخیر های ۰و۱و۲و۳و۱و۱ تحویل میدم؛ پس سرجمع هشت روز تاخیر کردم و نمره‌ام بدون کسر وارد میشه.
		\item
		 حالا اگر من به شکل ۷و۷و۱و۰و۰و۰ تاخیر کنم، تا تمرین سوم نمره‌ی کامل رو می‌گیرم و در تمرین سوم به علت تاخیر یک روزه، نصف نمره رو می‌گیرم؛ بقیه تمرین‌ها رو هم که کامل می‌گیرم.
		 \item و در سناریوی پایانی، تاخیرهای من رو به شکل ۱۴و۰و۱و۰و۲و۰
		 در نظر بگیرید؛ نمره‌ی تمرین اول و دوم کامله، تمرین سوم نصف میشه، تمرین چهارم و ششم کامله و نمره‌ی تمرین پنجم کاملا از دست میره.
	\end{itemize}
	\item
	لطفا در زمان میان‌ترم ها از ذخیره‌ی تاخیرهای مجازتون استفاده کنید و اجازه بدید که با زمان‌بندی تمرین‌ها پیش بریم.

	\item 
	تا حداکثر سه روز پس از اعلام‌ نمره‌ی تمرین فرصت دارید که اعتراضتون رو اعلام کنید.
\end{itemize}
\subsection*{نحوه‌ی تحویل تمرینات و دریافت اطلاعیه‌های درس}
همان‌طور که استاد در جلسه‌ی اولِ کلاس اعلام کردند، صفحه‌ی رسمی درس برای همه‌ی اطلاعیه‌ها و تمرین‌ها صفحه‌ی درس در
\href{https://cw.sharif.edu/}{ درس افزار شریف }
است و انتظار میره که همه‌ی دانشجویان صفحه درس رو چک کنند. 
\subsection*{موارد مربوط به تصحیح برگه‌ها}
\begin{enumerate}
	\item 
	تصحیح برگه‌های در ظرف یک هفته پس از پایان مهلت تمرین صورت می‌گیره.
	\item 
	درمورد برگه‌هایی که با تاخیر ارسال شدند (چه تاخیر آزاد چه جریمه‌دار)، لطفا پس از آپلود حل اون سری تمرین به ما اعلام کنید که تمرین را آپلود کردید تا ما در سری‌های بعدی (به همراه برگه‌های سایر دوستاتون) تصحیح ‌کنیم؛ چون متاسفانه سامانه، آپلودهای جدید رو به ما اعلام نمی‌کنه.
\end{enumerate}



\subsection*{درباره‌ی کلاسهای حل تمرین}
\begin{enumerate}
	\item
فعالیت کلاسی بخشی از ۶ نمره‌ی شماست. بنابراین برای کسب کردنش باید یک جلسه در کل جلسات حل تمرین، حضور «فعال» داشته باشید و به ما در برگزاری جلسه کمک کنید. کارهایی مثل جواب دادن به سوال و آمدن پای تخته ازنظر ما مشارکت فعالانه‌تون رو می‌رسونه.

\item
با توجه به پیشروی کلاس، تعداد تمرین‌ها متفاوته و بین هفتگی و دوهفتگی متغیره. هفته‌هایی که تمرین نداریم ممکن هست در کلاس کوئیز داشته باشیم؛ البته کوئیزها از قبل اعلام میشن.
\end{enumerate}
\subsection*{زمان‌بندی کلاس‌های حل تمرین}
دوتا کلاس حل تمرین داریم به شرح زیر:

\begin{enumerate}
\item
زهرا کبیری - 
\textbf{یکشنبه‌ها ۱۶:۳۰ تا ۱۸ }
-
         ابن سینا، الف۱۴
\vspace{-0.5em}
\item
حسین محمدی -
\textbf{دوشنبه‌ها ۹ تا ۱۰:۳۰ }
  - دانشکده فیزیک، ف۲
\end{enumerate}
\noindent
هریک از اعضای کلاس، لازم هست که در یکی از این کلاس‌ها ثبت نام کنند.

\vspace{2em}
\begin{boxes}{red}{نکات مهم}
	
	\begin{itemize}
		
		\item لیست‌گیری رو لطفا همه‌ی دانشجویان (چه کسانی که در کلاس حاضر میشن و چه کسانی که امکانش رو ندارند) انجام بدهند. هدف از لیست‌گیری برای تسهیل نمره‌دهی و تصحیح تمارین و بررسی آسون‌تر اعتراضاته.
		
		\item 
		حداقل نمره کسب شده در کلاس حل تمرین باید ۲ باشه؛ آنطور که استاد تاکید کردند، کسی که این حد نصاب رو کسب نکنه، نمره‌ی پایان و میان‌ترمش صفر وارد میشه.
		\item تاکید مجدد می‌کنم کسانی که توی هیچ یک از دو کلاس نمی‌توانند شرکت کنند حتما به استاد مراجعه کنند تا دلیلش رو با ایشون به بحث بگذارند.
	\end{itemize}
\end{boxes} 

\begin{boxes}{green}{سعی می‌کنیم اما قول نمی‌دهیم که:}
	
	\begin{itemize}
		
		\item یادداشت‌هایی از مطالبی که سرکلاس حل تمرین می‌گوییم با همه به اشتراک بگذاریم
 ، تا کسانی که امکان حضور سرکلاس ندارند بهره ببرند. (البته با اندکی تاخیر)		
		\item استاد از مطالب اضافه تر در کلاس حل تمرین استقبال کردند و قصد داریم در بعضی جلسات مطالب جدید و نسبتا پیشرفته‌تر بگوییم؛ مشروط به این‌که ببینم پیشروی با درس به نحو مطلوبی انجام میشه.
	\end{itemize}
\end{boxes} 





\end{document}