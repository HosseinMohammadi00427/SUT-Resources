\documentclass[a4paper, 12pt]{article}
\usepackage{physics, amsmath, amsfonts, fixmath, geometry, tikz, pgf, multirow, hyperref, amsfonts,amssymb, mathtools, physics, xcolor, siunitx, subcaption,tcolorbox}


\pdfpagewidth 8.5in
\pdfpageheight 11in
\headheight 0pt
\headsep 0pt
\footskip .25in
\marginparwidth 0pt
\marginparsep 0pt
\oddsidemargin \dimexpr 1in -1in
\topmargin \dimexpr 1in -1in
\textwidth \dimexpr \pdfpagewidth -2\oddsidemargin -2in
\textheight \dimexpr \pdfpageheight -2\topmargin -2in
%Commands

\newtcolorbox{boxes}[3][]
{
	colframe = #2!25,
	colback  = #2!10,
	coltitle = #2!40!black,  
	title    = {\textbf{#3}},
	#1,
}
\usepackage{graphicx}
\usepackage{xepersian}
\usepackage{braket}
\settextfont[]{XB Niloofar}
\title{\textbf{
 اصلاحات در روند کاری ما
}}
\author{درس نظریه گروه‌ها}
\date{}

\begin{document}
\maketitle
سلام.

\noindent

مطابق آنچه دیدیم و شنیدیم، کمی روند کارمان را تعدیل می‌کنیم. این فایل اصلاحیه‌ای است به فایلی که اول ترم تدوین کردیم و قرار است برخی شرایط را که از دید شما سخت‌گیرانه بود، تسهیل کنیم.

\subsection*{قوانین مربوط به تاخیر }
		\begin{itemize}
	\item 
	«فرصت تاخیر آزاد» از ۱۴ روز به ۲۰ روز افزایش پیدا می‌کنه.
	\item
	کسر نمره پس از استفاده از این ۲۰ روز، به این شکل تغییر می‌کنه: پایان روز اول یک‌چهارم نمره، پایان روز دوم نصف نمره، پایان روز سوم سه‌چهارم نمره و از ابتدای روز پنجم، نمره‌ کاملا از دست میره.
	
\end{itemize}

\subsection*{موارد مربوط به تصحیح برگه‌ها}
\begin{itemize}
	\item 
	تصحیح برگه‌ها دقیقا از پایان‌ هفته‌ی دوم پس از ددلاین شروع میشه. لطفا توجه کنید که اگرچه همچنان فرصت تحویل آزاد دارید؛ اما تعلق گرفتن نمره به تمرین‌شما به این معنی‌ است که دیگر فرصت اصلاح فایل شما وجود ندارد؛ پس تا از پاسخ‌هاتون مطمئن نشدید، فایل رو روی سامانه بارگذاری نکنید و از فرصت تاخیرتون استفاده کنید.

\end{itemize}



\subsection*{درباره‌ی کلاسهای حل تمرین}
\begin{itemize}
	\item
	فایل جلسات حل‌تمرین بلافاصله پس از پایان کلاس روی سامانه قرار می‌گیره.
\end{itemize}

\subsection*{موارد کلی مربوط به تمرینات}
\begin{itemize}
	\item
	مهلت تحویل نوعی تمرینات همچنان یک‌هفته خواهد بود.
	\item
	اما تعداد سوالات تا حد چهار سوال (بدون بخش‌بندی) یا دو الی سه سوال با بخش‌های مختلف خواهد بود. همچنان فکر می‌کنیم که هفتگی درگیر شدن با درس بسیار مناسب تر از دوهفتگی است.
	
	\item 
	در مورد سطح تمرینات هم اصلاحی انجام می‌گیره. سعی می‌کنیم تعداد تمرینات دشوار که نیاز به یادگرفتن مطالب فراتر از سطح کلاس داره، در حد یکی بماند.
\end{itemize}



\end{document}