\documentclass[a4paper, 12pt]{article}
\usepackage{physics, amsmath, amsfonts, fixmath, geometry, tikz, pgf, multirow, hyperref, amsfonts,amssymb, mathtools, physics, xcolor, siunitx, subcaption,tcolorbox}


\usepackage{graphicx}
\usepackage{braket}
\usepackage{enumitem,hyperref,dsfont}
\usepackage[linewidth= 2pt]{mdframed}
\usepackage{colortbl}
\usepackage{xepersian}
\setdigitfont{XB Niloofar}
\settextfont[]{XB Niloofar}
\deflatinfont\enfont[Scale=1]{Times New Roman}
\newcommand{\pcur}[0]{\lr{Page curve}}
\newcommand{\mycomment}[1]{}
\title{\textbf{
  زیرگروه‌های بهنجار گروه 
  $S_4$
}}
\author{حسین محمدی
\\
زهرا کبیری}
\date{}
\newtcolorbox{boxes}[3][]
{
	colframe = #2!25,
	colback  = #2!10,
	coltitle = #2!40!black,  
	title    = {\textbf{#3}},
	#1,
}

\begin{document}
\maketitle

\begin{mdframed}
	به کمک آقای اشتری فهمیدم که یک تمرین‌ را برای شما اشتباه حل کردم. حالا اینجا سعی می‌کنم آن را دقیق و کامل بررسی کنم.
\end{mdframed}

هدف دسته‌بندی تمام زیرگروه‌های بهنجار گروه $S_4$ است.

در کلاس دیدید که یک زیرگروه بهنجار است اگر و تنها اگر تمامی اعضای یک کلاس‌ تزویجی را دربربگیرد. مطابق رابطه‌ی 
\[
\sigma (i_1 i_2 \dots i_r)\sigma^{-1} = (\sigma(i_1) \sigma(i_2) \dots \sigma(i_r))
\]
که در آن 
$\sigma \in S_n$
 است؛ می‌توانیم ببینم که عمل همیوغ کردن، ساختار دوری را عوض نمی‌کند. بنابراین، کلاس‌های تزویجی یک گروه، در حقیقت متشکل از دورهای با ساختار دوری مشخص هستند. بیایید کلاس‌های تزویجی گروه $S_4$ را پیدا کنیم.
\begin{enumerate}
	\item 
	دورهای به طول ۴ که ساختار 
	$(ijkl)$ دارند. توجه کنید که دقیقا ۶ عنصر از 
	$S_4$
	دور به طول چهار هستند.
	\[
	C_1=\{(1234),(1243),(1324),(1342),(1423),(1432)\}
	\]
	\item دورهایی که طول سه دارند و به شکل 
	$(ijk)(l)$
	هستند
	\footnote{مقصود از $(l)$ جایگشتی است که هر عضو را به خودش می‌برد و ما معمولا در نوشته‌هایمان آن را نمی‌نویسیم.}.
	این کلاس تزویجی ۸ عضو دارد.
	\begin{equation*}
		\begin{aligned}
			C_2 = \{
			(123),(132),(124),(142),(234),(243),(134),(143)
			\}
		\end{aligned}
	\end{equation*}
	\item سومین کلاس تزویجی مربوط به حاصل‌ضرب دو ترانهش مجزاست؛ این کلاس هم شامل ۳ عضو است.
	\[
	C_3= \{(12)(34),(13)(24),(14)(23))\}
	\]
	\item همچنین ترانهش های تکی هم یک کلاس تزویجی هستند که شش عضو دارد.
	\[
	C_4 =\{(12),(13),(14),(23),(24),(34)\}
	\]
	\item آخرین کلاس هم کلاس تزویجی بدیهی است که شامل عضو بدیهی است.
	\[
	C_5 = \{e\}
	\]
\end{enumerate}

حالا احتمالا به یاد دارید که ترانهش‌های تکی کل گروه را می‌سازند؛ بنابراین درزیرگروه بهنجارمان (که باید تمامی اعضای کلاس تزویجی را شامل شود) نمی‌توان ترانهش تکی داشت؛ چون اگر تنها یک ترانهش در زیرگروه $H$ باشد، با ضرب ترانهش ها می‌توان کل گروه را ساخت.
($H=S_4$)

مطابق قضیه لاگرانژ، مرتبه‌ی هر زیرگروهی (از جمله زیرگروه‌های بهنجار) باید مرتبه‌ی گروه اصلی را بشمارد. این یعنی مرتبه‌ی هر زیرگروهی باید 1 یا ۲ یا ۳ یا ۴ یا ۶ یا ۱۲ باشد؛ زیرگروه مرتبه ۲۴ دقیقا خود گروه $S_4$ است که در خودش به شکلی بدیهی بهنجار است.

همچنین به خاطر قضیه‌ای که در کلاس دیدید؛ مرتبه‌ی زیرگروه بهنجار باید جمع تعداد اعضای کلاس‌های تزویجی باشد که در آن حاضرند. یعنی مرتبه‌ی زیرگروه بهنجار باید از جمع ۱و۳و۶و۸ حاصل شود و همچنین چون عضو بدیهی در هرگروهی هست، باید ۱ حتما در این جمع باشد.

\begin{enumerate}
	\item یک حالت این است که فقط کلاس تزویجی بدیهی را در $H$ بگنجانیم؛ در این حالت زیرگروهی بدیهی را خواهیم داشت که به وضوح در $S_4$ بهنجار است.
	\item
	حالت دیگر این است که کلاس بدیهی به همراه کلاس تزویجی $C_3$ را در زیرگروه $H$ بگنجانیم. در این صورت مرتبه‌ی گروه ۴ است و همچنین خواص گروه را هم خواهد داشت (چرا؟).
	\[H = \{e , (12)(34),(13)(24),(14)(23)\}\]
	\item حالت دیگر این است که کلاس بدیهی، کلاس 
	$C_2$ به همراه کلاس 
	$C_1$ را در نظر بگیریم. 
	در این صورت زیرگروه حاصل ۱۲ عضوی خواهد شد. این زیرگروه دقیقا همان زیرگروه 
	$A_4$ بدست می‌آید که هسته‌ی نگاشت 
	$\text{sgn}$
	است که در تمرین‌ها با آن آشنا شدیم.
	\item  در حالت آخر هم اگر تمامی کلاس‌های تزویجی را در نظر بگیریم، کل گروه حاصل می‌شود و به شکلی بدیهی
	$S_4$ در خودش بهنجار است.
\end{enumerate}
 پس تعداد زیرگروه‌های بهنجار گروه 
 $S_4$
 دقیقا چهارتاست؛ خودش، زیرگروه بدیهی، زیرگروه 
 $A_4$ و همچنین زیرگروه 
 $H$ با تعریف بالا.

\end{document}