
\documentclass[12pt]{article}
\parindent 0mm
\usepackage[a4paper,
bindingoffset=0.2in,
left=1in,
right=1in,
top=0.5in,
bottom=0.8in,
footskip=.25in]{geometry}
\usepackage{amsmath,amsthm,amssymb,diagbox,physics, amsfonts, fixmath, tikz, pgf, multirow, hyperref, amsfonts, mathtools, physics, xcolor, siunitx, subcaption,tcolorbox}

\usepackage{xepersian}
\settextfont[]{XB Niloofar}
\setdigitfont{XB Niloofar}
\newtcolorbox{boxes}[3][]
{
	colframe = #2!25,
	colback  = #2!10,
	coltitle = #2!40!black,  
	title    = {\textbf{#3}},
	#1,
}
	
	\newenvironment{exercise}[3][\unskip]{%
	\par
	\noindent
	\textbf{
		 #1
		 [#2 امتیاز] 
		 \def\temp{#3}\ifx\temp\empty
		 : 
		 \else
		 : #3 \vspace{0.5em} \\
		 \fi
		}}{}

\begin{document}

	
	\title{آزمونک چهارم - درس نظریه‌ گروه‌ها
		\\
	{\normalsize 
	\textbf{استاد درس:}
	دکتر رضاخانی
	}
	}
	\author{
	\small
	\textbf{دستیارهای درس:}
	 حسین محمدی، زهرا کبیری
	} 
\date{۳۱ اردیبهشت‌ماه سال ۱۴۰۳}
	\maketitle
	سوال‌های زیر با هدف مرور مفاهیم اولیه گروه‌های ماتریسی طراحی شده‌است.
	
	از بین دو سوال زیر یکی را به دلخواه انتخاب و حل کنید.

\vspace{0.4em}
\par\noindent\rule{\textwidth}{0.4pt}

\vspace{0.4em}
\noindent
	\textbf{ سوال (۱):}
$G$ را گروه ماتریسی 
زیر بگیرید.
\[
G = \Biggl\{
\begin{pmatrix}
	a&b\\c&d
\end{pmatrix} \; \Big| \; a,b,c,d \in \mathbb{Z}
\Biggr\}
\]
عملِ گروه هم جمع اعداد صحیح است. زیرگروه $H$ را متشکل از ماتریس‌های 
۲در۲ بگیرید که تمام درایه‌هایش زوج است. گروه خارج‌قسمتی 
$G/H$
را پیدا کنید و مرتبه‌اش را بنویسید.
	

	
	
\vspace{0.4em}
	\par\noindent\rule{\textwidth}{0.4pt}
	
		\vspace{0.4em}
		\noindent
	\textbf{ سوال (۲):}
	
	ماتریس‌های 
	$R \in \text{SO}(3)$
	و 
	$L \in \text{SL}(2,\mathbb{R})$
	و ماتریس دلخواه 
	$T$
	که 
	$3\times2$
	است و درایه‌های حقیقی دارد، در نظر داشته باشید.
 همچنین ماتریس 
 $0_{2\times 3}$
 ماتریسی است که تمام درایه‌هایش صفر است.
 
 ماتریس 
 $5 \times 5$
 زیر را بسازید.
 \[
  \left[
 \begin{array}{c|c}
 	R_{\small 3\times 3} & T_{\small 3\times 2}\\
 	\hline
 	0_{\small 2\times 3} & L_{\small 2\times 2}
 \end{array}
 \right]
 \]	
 آیا این ماتریس‌ها با عمل ضرب ماتریسی، تشکیل یک گروه می‌دهند؟
	
	
	
	
	
	
	
	
\end{document}