
\documentclass[12pt]{article}

\usepackage[a4paper,
bindingoffset=0.2in,
left=1in,
right=1in,
top=0.5in,
bottom=0.8in,
footskip=.25in]{geometry}
\usepackage{amsmath,amsthm,amssymb,diagbox,physics, amsfonts, fixmath, tikz, pgf, multirow, hyperref, amsfonts, mathtools, physics, xcolor, siunitx, subcaption,tcolorbox}

\usepackage{xepersian}
\settextfont[]{XB Niloofar}
\setdigitfont{XB Niloofar}
\newtcolorbox{boxes}[3][]
{
	colframe = #2!25,
	colback  = #2!10,
	coltitle = #2!40!black,  
	title    = {\textbf{#3}},
	#1,
}

\newenvironment{exercise}[3][\unskip]{%
	\par
	\noindent
	\textbf{
		#1
		[#2 امتیاز] 
		\def\temp{#3}\ifx\temp\empty
		: 
		\else
		: #3 \vspace{0.5em} \\
		\fi
}}{}

\begin{document}
	
	
	\title{آزمونک سوم - درس نظریه‌ گروه‌ها
		\\
		{\normalsize 
			\textbf{استاد درس:}
			دکتر رضاخانی
		}
	}
	\author{
		\small
		\textbf{دستیارهای درس:}
		حسین محمدی، زهرا کبیری
	} 
	\date{10 اردیبهشت ماه سال 1403}
	\maketitle
	سوال زیر با هدف مرور مفاهیم اثر گروه روی مجموعه طراحی شده است.
	
	\vspace{1.5em}
	\noindent
	\textbf{ سوال:}
	مجموعه‌ی 
	\[
	G = \Biggl\{
	\begin{bmatrix}
		1&0 \\ 0&1
	\end{bmatrix},
	\begin{bmatrix}
		-1&0 \\ 0&1
	\end{bmatrix},
	\begin{bmatrix}
		1&0 \\ 0&-1
	\end{bmatrix},
	\begin{bmatrix}
		-1&0 \\ 0&-1
	\end{bmatrix},
	\begin{bmatrix}
		0&1 \\ 1&0
	\end{bmatrix},
	\begin{bmatrix}
		0&1 \\ -1&0
	\end{bmatrix},
	\begin{bmatrix}
		0&-1 \\ 1&0
	\end{bmatrix},
	\begin{bmatrix}
		0&-1 \\ -1&0
	\end{bmatrix}
	\Biggr\}
	\]
	را با عمل ضرب ماتریسی در نظربگیرید. این مجموعه یک زیرگروه متناهی از گروه
	$\text{GL}_2(\mathbb{R})$
	را تشکیل می‌دهد.
	
	\noindent
	همانطور که می‌دانیم؛ این گروه با ضرب ماتریسی روی بردارهای صفحه عمل می‌کند.
	
	\vspace{1.5em}
	\noindent
	مدارهای اثر این گروه روی بردارها را دسته بندی کنید. یعنی اثر اعضای $G$ روی اعضای
\[
e_1 = 	\begin{bmatrix}
		0 \\ 0
	\end{bmatrix},
e_2 = 	\begin{bmatrix}
		x \\ x
	\end{bmatrix},
	e_3 = \begin{bmatrix}
		0 \\ y
	\end{bmatrix},
e_4 = 	\begin{bmatrix}
		x \\ y
	\end{bmatrix}
\]
	با شرط
	(
	$x\neq y$ و $x,y\neq 0$
	)
	بدست آورید و سپس مجموعه
	$O_G(e_i)$
	را برای بردارهای بالا بسازید.
	
	\noindent
	استدلال کنید که مدارها، مجموعه‌ی بردارهای دوبعدی را افراز می‌کنند.
	
	
	
	
	
	
	
	
	
\end{document}