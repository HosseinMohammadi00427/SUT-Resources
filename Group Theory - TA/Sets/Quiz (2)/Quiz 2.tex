
\documentclass[12pt]{article}

\usepackage[a4paper,
bindingoffset=0.2in,
left=1in,
right=1in,
top=0.5in,
bottom=0.8in,
footskip=.25in]{geometry}
\usepackage{amsmath,amsthm,amssymb,diagbox,physics, amsfonts, fixmath, tikz, pgf, multirow, hyperref, amsfonts, mathtools, physics, xcolor, siunitx, subcaption,tcolorbox}

\usepackage{xepersian}
\settextfont[]{XB Niloofar}
\setdigitfont{XB Niloofar}
\newtcolorbox{boxes}[3][]
{
	colframe = #2!25,
	colback  = #2!10,
	coltitle = #2!40!black,  
	title    = {\textbf{#3}},
	#1,
}
	
	\newenvironment{exercise}[3][\unskip]{%
	\par
	\noindent
	\textbf{
		 #1
		 [#2 امتیاز] 
		 \def\temp{#3}\ifx\temp\empty
		 : 
		 \else
		 : #3 \vspace{0.5em} \\
		 \fi
		}}{}

\begin{document}

	
	\title{آزمونک دوم - درس نظریه‌ گروه‌ها
		\\
	{\normalsize 
	\textbf{استاد درس:}
	دکتر رضاخانی
	}
	}
	\author{
	\small
	\textbf{دستیارهای درس:}
	 حسین محمدی، زهرا کبیری
	} 
\date{20 فروردین ماه سال 1403}
	\maketitle
	سوال زیر با هدف مرور مفاهیم اولیه و با تاکید بیشتر روی هم‌مجموعه‌ها و زیرگروه‌های بهنجار طراحی شده است.
	
	\vspace{1.5em}
	\noindent
	\textbf{ سوال:}
 گروه 
 $D_6$  را در نظر بگیرید.
 \[
 D_6 = \{1,x,x^2,x^3,x^4,x^5,y,xy,x^2y,x^3y,x^4y,x^5y \;\big|\; x^6=1 \; , \; y^2=1 \;,\; yx = x^5y\}
 \]
 
 \noindent
	الف) زیرگروه‌های از مرتبه ۲، ۳ و ۶ را مشخص کنید.
	 
	(راهنمایی: ۷ زیرگروه مرتبه ۲، یک زیرگروه مرتبه 3 و یک زیرگروه مرتبه ۶ داریم.)
	
	\noindent
	ب) همدسته‌های چپ زیرگروه مرتبه‌ی ۳ ،$H$،را بسازید؛ مطابق  رابطه‌ی اندیس گروه می‌دانید که 
	\[[D_6:H] = 4, \]
 پس باید ۴ همدسته‌ی سه عضوی پیدا کنید.
	
\noindent
ج) حالا همدسته‌های راست زیرگروه $H$ را بیابید. بازهم باید چهار همدسته‌ی سه عضوی پیدا کنید.

\noindent
د)تعریفی از بهنجاری برحسب همدسته‌ها وجود دارد؛ آن تعریف را به خاطر آورید و معین کنید که آیا زیرگروه $H$ در گروه $D_6$ بهنجار است؟

\noindent
ه) زیرگروه چهارعضوی 
$I = \{1,x^3,y,x^3y\}$
را در نظربگیرید و مراحل (ب) و (ج) و (د) را برای این زیرگروه تکرار کنید.

\noindent
و) نشان دهید ضرب همدسته‌ها با تعریف 
\[
xI . x^2I \overset{?}{=}(x.x^2) I = x^3 I
\]
	برقرار نیست. (راهنمایی: عضوی از ضرب درونی 
	$xI . x^2I$
	بیابید که در 
	$x^3I$
	نباشد.
	)
	
	\noindent
	این یکی از آزمون‌هایی است که نشان می‌دهد زیرگروه $I$ بهنجار نیست.
	

	
	

	
	
	
	
	
	
	
	
	
	
\end{document}