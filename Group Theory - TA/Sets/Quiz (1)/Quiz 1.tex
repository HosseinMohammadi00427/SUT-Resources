
\documentclass[12pt]{article}

\usepackage[a4paper,
bindingoffset=0.2in,
left=1in,
right=1in,
top=0.5in,
bottom=0.8in,
footskip=.25in]{geometry}
\usepackage{amsmath,amsthm,amssymb,diagbox,physics, amsfonts, fixmath, tikz, pgf, multirow, hyperref, amsfonts, mathtools, physics, xcolor, siunitx, subcaption,tcolorbox}

\usepackage{xepersian}
\settextfont[]{XB Niloofar}
\newtcolorbox{boxes}[3][]
{
	colframe = #2!25,
	colback  = #2!10,
	coltitle = #2!40!black,  
	title    = {\textbf{#3}},
	#1,
}
	
	\newenvironment{exercise}[3][\unskip]{%
	\par
	\noindent
	\textbf{
		 #1
		 [#2 امتیاز] 
		 \def\temp{#3}\ifx\temp\empty
		 : 
		 \else
		 : #3 \vspace{0.5em} \\
		 \fi
		}}{}

\begin{document}

	
	\title{آزمونک اول - درس نظریه‌ گروه‌ها
		\\
	{\normalsize 
	\textbf{استاد درس:}
	دکتر رضاخانی
	}
	}
	\author{
	\small
	\textbf{دستیارهای درس:}
	 حسین محمدی، زهرا کبیری
	} 
\date{۱۴ اسفند سال ۱۴۰۲}
	\maketitle
	از بین دو سوال زیر، به دلخواه یکی را انتخاب و حل کنید.
	
	\vspace{1.5em}
	\noindent
	\textbf{ سوال اول:}
 ثابت کنید اگر به ازای هر عضو 
 $a,b \in G$
 از گروه متناهی 
 $G$،
  داشته باشیم 
  $(a.b)^2 = a^2.b^2$، آنگاه
  گروه G آبلی است.
	
	\vspace{1.5em}
	\noindent
	\textbf{ سوال دوم:}
	مجموعه‌ی 
	$G$ را مجموعه‌ی ماتریس‌های
	۲در۲، متشکل از درایه‌های صفر و یک درنظر بگیرید که دترمینان‌شان به پیمانه‌ی ۲ ناصفر است.
	\[
	G= \{
	\begin{pmatrix}
		a & b \\
		c & d \\
	\end{pmatrix}
	\Big| \;a,b,c,d \in \{0,1\}
	\;\; ,\;\; ad-bc \neq 0 \;\;(\text{mod} \; 2)
	\}
	\]
	نشان دهید که تحت عمل ضرب ماتریسی (به پیمانه‌ی ۲) این مجموعه تشکیل گروه می‌دهد؛ ضمنا تمامی اعضای گروه را مشخص کنید.
	

	
	
	
	
	
	

	
	

	
	
	
	
	
	
	
	
	
	
\end{document}