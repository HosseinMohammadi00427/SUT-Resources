\documentclass{article}
\usepackage[final]{neurips}
\usepackage[framemethod=tikz]{mdframed}
\usepackage{lipsum}
\definecolor{mycolor}{rgb}{0.122, 0.435, 0.698}
\newmdenv[innerlinewidth=0.5pt, roundcorner=4pt,linecolor=mycolor,innerleftmargin=6pt,
innerrightmargin=6pt,innertopmargin=6pt,innerbottommargin=6pt]{mybox}
\usepackage[utf8]{inputenc} % allow utf-8 input
\usepackage[T1]{fontenc}    % use 8-bit T1 fonts
\usepackage[hidelinks]{hyperref}       % hyperlinks
\usepackage{url}            % simple URL typesetting
\usepackage{booktabs}       % professional-quality tables
\usepackage{amsfonts}       % blackboard math symbols
\usepackage{nicefrac,tcolorbox}       % compact symbols for 1/2, etc.
\usepackage{amsmath}
\usepackage{enumitem}
\usepackage{microtype}      % microtypography
\usepackage{graphicx,caption}
\usepackage{xepersian}
\settextfont{XB Niloofar}
\setdigitfont{XB Niloofar}
\raggedbottom


\title{
	\vspace{-0.8em}
	تمرین سری هفتم درس نظریه گروه‌ها - دکتر رضاخانی
	\\
	{\normalsize
		\textbf{مهلت تحویل:
			چهار‌‌شنبه ۲ خرداد ماه سال ۱۴۰۳ تا ساعت 59:23
			\\
			\vspace{-0.4em}
			از طریق سامانه
			\href{https://cw.sharif.edu/}{درس‌افزار شریف}
		}
	}
	\vspace{-0.6em}
}

\usepackage[utf8]{inputenc}

\usepackage[english]{babel}
\setlength{\parindent}{3.5em}
\setlength{\parskip}{0.5em}
\renewcommand{\baselinestretch}{1.0}

\usepackage{calrsfs}
\DeclareMathAlphabet{\pazocal}{OMS}{zplm}{m}{n}
\newcommand{\La}{\mathcal{L}}
\newcommand{\Lb}{\pazocal{L}}

\newtcolorbox{boxes}[3][]
{
	colframe = #2!25,
	colback  = #2!10,
	coltitle = #2!40!black,  
	title    = {\textbf{#3}},
	#1,
}

\newenvironment{exercise}[3][\unskip]{%
	\par
	\noindent
	\textbf{تمرین
		#1
		[#2 امتیاز] 
		\def\temp{#3}\ifx\temp\empty
		: 
		\else
		: #3 \vspace{0.5em} \\ \noindent
		\fi
}}{}

\author{
	حسین محمدی\\
	\lr{
		\href{mailto:hossein.mohammadi.00427@gmail.com}{\texttt{	hossein.mohammadi.00427@gmail.com}}} \\
	\And
	زهرا کبیری\\
	\lr{
		\href{mailto:kabiri.zahra98@gmail.com}{ \texttt{kabiri.zahra98@gmail.com}}}\\
}

\begin{document}
	
	
	\begin{minipage}{0.1\textwidth}% adapt widths of minipages to your needs
		\includegraphics[width=1.1cm]{sharif-logo.png}
	\end{minipage}%
	\hfill%
	\begin{minipage}{0.9\textwidth}\raggedleft
		دانشگاه صنعتی شریف\\
		زمستان ۱۴۰۲ - بهار ۱۴۰۳\\
	\end{minipage}
	
	\makepertitle
	
	
	\begin{exercise}[29]{40}{زیرگروه ویژه‌ی لورنتز}
		زیرمجموعه ای از
		$O(1,3)$
		که در زیر معرفی شده است، زیرگروه ویژه لورنتز نام دارد.
		\[
		O_{\uparrow}^{+}(1,3) = \{\Lambda \in O(1,3) | \det (\Lambda) = 1 , \Lambda_{00} \geq 1 \}
		\]
		نشان دهید که این زیرمجوعه یک زیرگروه از 
		$O(1,3)$
		است و همچنین این زیرگروه بهنجار است.
		
	\end{exercise}
	
	
	\begin{exercise}[30]{30}{گروه 
			$\text{SL}(2,\mathbb{Z})$
		}
		ساده‌ترین گروه گسسته، متناهی و غیرآبلی، گروه $\text{SL}(2,\mathbb{Z})$
		است. این گروه متشکل از ماتریس‌های ۲ در ۲ با دترمینان یک است که درایه‌هایش اعداد صحیح هستند. نامِ ویژه‌ی 
		\lr{Modular group}
		به علت اهمیت فراوان این گروه در تبدیلات خشتی چنبره
		\LTRfootnote{Modular trnasformations of torus}
		داده شده است؛ بعدها که نظریه ریسمان یا نظریه میدان همدیس آشنا شدید، به این گروه و اهمیتش در فیزیک بیشتر پی‌ خواهید برد. اما چیزی که از شما می‌خواهیم خیلی ساده است:
		
		\noindent
		نشان دهید که این گروه با دو عضو 
		\[
		S = \begin{pmatrix}
			0 & -1 \\ 1 & 0
		\end{pmatrix} , \qquad T =  \begin{pmatrix}
			1 & 1 \\ 0 & 1
		\end{pmatrix}
		\]
		تولید می‌شود؛ یعنی هر عضو دلخواهی از این گروه را به شکل ضرب این دو عضو بنویسید.
		
	\end{exercise}
	
	\begin{exercise}[31]{30}{}
		یک نوسانگر هماهنگ یک بعدی داخل میدان الکتریکی 
		$E$ 
		در نظر بگیرید.
		\begin{equation*}
			H=\frac{P^2}{2m}+\frac{1}{2}m\omega ^2 x^2 - eEx
		\end{equation*}
		که در آن 
		$m$
		جرم، 
		$\omega$ 
		بسامد، 
		$e$ 
		بار الکتریکی نوسانگر است.
		\\
		الف) تبدیل خطی از 
		$x$ 
		پیدا کنید که پتانسیل 
		$V=\frac{1}{2}m\omega ^2 x^2 - eEx$ 
		را به 
		$\Tilde{V}=\frac{1}{2}m\omega ^2 \Tilde{x}^2 +V_0$ 
		برساند.\\
		ب) نشان دهید این تبدیل یک تبدیل هم تافته 
		\LTRfootnote{Symplectic}
		است.\\
		ج) دوباره به نوسانگر هماهنگ ساده بر می گردیم. فرض کنید کمیت های 
		$m$ 
		و 
		$\omega$ 
		به گونه ای است که همیلتونی آن چنین است:
		\begin{equation*}
			H=\frac{1}{2}(P^2+x^2)
		\end{equation*}
		بررسی کنید آیا تبدیل زیر یک تبدیل هم تافته است؟
		\begin{equation*}
			\Tilde{x}=\frac{1+i}{2}(x+iP) \quad , \quad \Tilde{P}=\frac{1+i}{2}(x-iP)
		\end{equation*}
	\end{exercise}
	
	\begin{boxes}{black}{معرفی تعمیمی از گروه‌های متعامد}
		با گروه‌های متعامد
		\LTRfootnote{Orthogonal groups}
		در درس آشنا شده‌اید؛ 
		\href{https://en.wikipedia.org/wiki/Indefinite_orthogonal_group}{تعمیمی}
		از این گروه‌ها وجود دارند که از قضا در فیزیک این تعمیم‌ها جالب توجه هستند.
		
		اول دو عدد طبیعی $p$ و $q$ را در نظر بگیرید. ماتریس $g$ را که ماتریسی 
		$(p+q) \times (p+q)$
		است اینطور می‌سازیم: روی $p$ درایه‌ی اول قطر اصلی، عدد یک را قرار دهید و روی $q$ درایه‌ی بعدی، عدد ۱- را بگذارید؛ یعنی 
		\[
		g = \text{diag} (\underbrace{1,\dots,1}_{p},\underbrace{-1,\dots,-1}_{q})
		\]
		حالا آماده‌ایم که «گروه‌های متعامد غیرمعیّن» را تعریف کنیم
		\footnote{منظور از 
			$M_{p+q}(\mathbb{R})$
			، مجموعه‌ی تمامی ماتریس‌های 
			$(p+q) \times (p+q)$
			است که درایه‌های آن حقیقی هستند.
		}
		\footnote{از خانم حانیه ملکی برای یادآوردن شدن اشکالات این بخش متشکریم.}
		.
		\[
		O(p,q) = \{
		A \in M_{p+q}(\mathbb{R}) \; \big| \; A^t g A = g 
		\}
		\]
		برای یادآوری، $g$ به ازای $p=1$ و $q=3$ به متریک مینکوفسکی
		\LTRfootnote{Minkowski}
		تبدیل می‌شود و گروه 
		$O(1,3)$ 
		هم همان گروه لورنتز است. تعریف معادلِ این گروه اینطور است:  $O(p,q)$ شامل اعضایی است که ضرب برداری تعمیم‌یافته
		\footnote{
			منظورمان از ضرب برداری تعمیم یافته این است:
			\[
			\left< x,y \right> = x_1y_1 + \dots x_py_p - x_{p+1}y_{p+1} -\dots - x_{p+q} y_{p+q}
			\]
			که مشابه با ضرب چاربردارها از فضای میکوفسکی، با استفاده از متریک $g$ تعریف شده است.
		}
		با متریک $g$ را ناوردا نگه‌دارند؛ برای اطلاع بیشتر به 
		\href{https://en.wikipedia.org/wiki/Indefinite_orthogonal_group}{این صفحه از ویکی‌پدیا}
		مراجعه کنید.
		\[
		O(p,q) = \{
		A \in M_{p+q}(\mathbb{R}) \; \big| \; \left< Ax,Ay \right> = \left< x,y \right> \;\; , \;\; \forall x,y \in \mathbb{R}^{p+q}
		\}
		\]
	\end{boxes}
	
	
	\vspace{1em}
\end{document}
