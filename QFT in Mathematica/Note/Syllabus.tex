\documentclass[a4paper, 12pt]{article}
\usepackage{physics, amsmath, amsfonts, fixmath, geometry, tikz, pgf, multirow, hyperref, amsfonts,amssymb, mathtools, physics, xcolor, siunitx, subcaption,tcolorbox}


\usepackage{graphicx}
\usepackage{braket}
\usepackage{enumitem,hyperref,dsfont}
\usepackage[linewidth= 2pt]{mdframed}
\usepackage{colortbl}
\usepackage{xepersian}
\setdigitfont{XB Niloofar}
\settextfont[]{XB Niloofar}
\deflatinfont\enfont[Scale=1]{Times New Roman}
\newcommand{\pcur}[0]{\lr{Page curve}}
\newcommand{\mycomment}[1]{}
\title{\textbf{
توضیحات مربوط به پکیج‌های مربوط به نظریه میدان کوانتومی در 
\lr{Mathematica}
}}
\author{حسین محمدی
\\}
\newtcolorbox{boxes}[3][]
{
	colframe = #2!25,
	colback  = #2!10,
	coltitle = #2!40!black,  
	title    = {\textbf{#3}},
	#1,
}

\begin{document}
\maketitle

سلام.

در این فایل به طور خلاصه روندی را که برای ارائه در ذهن داشتم و همچنین فایل‌هایی را پیوست شده‌اند، توضیح می‌دهم.

من هفت جلسه‌ در نظر داشتم که حدود ۸ تا ۹ ساعتی طول می‌کشید. 
\begin{enumerate}
	\item 
	درجلسه‌ی اول مقدماتی برای کار با متمتیکا، معرفی پکیج‌ها، نصبشان و مثال‌های کمینه را معرفی می‌کردم.
	
	\item 
	جلسه‌ی دوم و سوم به بررسی هسته‌ی اصلیِ کار یعنی مقدماتِ 
	\lr{FeynCalc}
	می‌پرداخت. از چاربردارها گرفته تا اسپینورها و همچنین جبرهای 
	$SU(N)$
	برای نظریه میدان‌های غیرجابه‌جایی.
	
	\item 
	جلسه‌ی چهارم به 
	\lr{FeynArts}
	و ترسیم دیاگرام‌ها با مرتبه‌ی دلخواه جفتیدگی و همچنین نظریه میدان اختلالی می‌پرداخت.
	
	\item 
	جلسه‌ی پنجم به بررسی 
	\lr{FeynRules}
	می‌پرداخت؛ این‌که چطور از لاگرانژی به قوانین فاینمن برسیم. 
	
	\item 
	جلسات ششم و هفتم (و حتی هشتم!) به بررسی مدل‌های نظریه‌میدان و بازبهنجارششان می‌پرداخت.
	از 
	$\phi^3$
	گرفته تا نظریه‌های پیمانه‌ای  و 
	\lr{Electroweak}.
	این قسمت‌ها مقدار زیادی تکنیکی هستند و اگر نتوانیم با فهم درست از ماژول‌های مختلف 
	\lr{FeynCalc}
	استفاده کنیم؛ تقریبا نتیجه‌ای حاصل نمی‌شود.
\end{enumerate}


\begin{mdframed}
	پروژه‌ی نهایی هم پیاده‌سازی کامل و بازبهنجارش یک 
	\lr{MSSM}
	\LTRfootnote{Minimal Super symmetric Standard Model}
	مختلط بود.  می‌توانستند بازبهنجارش دوحلقه‌ی این مدل را هم ببینید.
\end{mdframed}


حالا فایل‌ها را به ترتیب توضیح دهم.
\begin{itemize}
	\item 
	فایل مقدمه که شامل توضیحات و نحوه‌ی نصب و مثال کمینه است به فرمت 
	\lr{nb}
	یعنی به فرمت دفترچه یادداشت‌های متمتیکا است.
	\item 
	فایل جلسات دوم و سوم یک 
	\lr{Tutorial}
	مقدماتی است که تمامی مباحث اساسی 
	\lr{FeynCalc}
	را توضیح می‌دهد. یادم هست که خودم بازبهنجارش فروسرخ مربوط به دیاگرام‌های درختی پراکندگی الکترون-پوزیترون را با کمک دستورهای این فایل  انجام دادم.
	
	\item جلسه‌ی چهارم و پنجم هم فایل 
	\lr{Documentation}
	پکیج
	\lr{FeynArts}
	و
	\lr{FeynRules}
	است؛ به نظرم این فایل خیلی مناسب و خوب مثال‌های اساسی را توضیح داده. اگر قرار باشد در آینده در این مورد ارائه‌ای بدهم، همین را دستور کار قررا می‌دهم.
	\item 
	فایل جلسات آخر، شامل تمامی مثالهاست. تئوری‌های 
	$\phi^3$
	و
	$\phi^4$
	و $QED$ و $QCD$ و همچنین الکتروضعیف را دربرمی‌گیرد. برای هر کدام دیاگرام‌های درختی، تک‌حلقه و دوحلقه و بازبهنجارشش هست.
	چیزی که مهم است این است که پکیج‌ها به درستی بارگذاری شوند و همچنین دستورها به ترتیب اجرا شوند.
\end{itemize} 


\end{document}