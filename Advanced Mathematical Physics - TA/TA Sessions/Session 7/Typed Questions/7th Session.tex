\documentclass[a4paper, 12pt]{article}
\usepackage[top=20mm, bottom=25mm, left=1.6in, right=1.6in,includehead=true,includefoot=true]{geometry}
\usepackage{physics, amsmath, amsfonts, fixmath, geometry, tikz, pgf, multirow, hyperref, amsfonts,amssymb, mathtools, physics, xcolor, siunitx, subcaption,tcolorbox}
\usepackage{graphicx}
\usepackage{braket}
\usepackage{enumitem,hyperref,dsfont}
\usepackage[linewidth= 1pt]{mdframed}
\usepackage{colortbl}
\usepackage{xepersian}
\setdigitfont{XB Niloofar}
\settextfont[]{XB Niloofar}
\deflatinfont\enfont[Scale=1]{Times New Roman}

\newcommand{\mycomment}[1]{}
\parindent 0mm
\newenvironment{parind}{%
	\par%
	\medskip
	\leftskip=0mm\rightskip=7mm
	\noindent\ignorespaces}{%
	\par\medskip}

\parindent 0mm
\title{\textbf{
 جلسه هفتم حل تمرین درس ریاضی‌فیزیک پیشرفته
 \\
 \small
 ‌هندسه ریمانی
}}
\author{حسین محمدی
\\
\small
چهارشنبه 26 اردیبهشت سال 1403
}
\date{}
\newtcolorbox{boxes}[3][]
{
	colframe = #2!25,
	colback  = #2!10,
	coltitle = #2!40!black,  
	title    = {\textbf{#3}},
	#1,
}

\begin{document}
\maketitle

\textbf{سوال اول:
درباب مقدمات خمینه‌های ریمانی
}

\vspace{0.7em}
خمینه‌ی ریمانی
$(M,g)$ یک خمینه‌ی 
$n$-بعدی است. نشان دهید که: 
\begin{parind}
	الف) به‌ازای هر 
	$\alpha,\beta \in T_p^\star M$
	و هرپایه‌ی یکه‌متعامد
	$\{e_i\}_{i=1}^n$
	از فضای مماسِ
	$T_pM$
	داریم:

	\[
	g^{-1}(\alpha,\beta) = \sum_{i} \alpha(e_i)\beta(e_i)
	\]
که در آن 
$g^{-1}$
متریک پادوردای مربوط به $g$ است؛ در کاربرد‌های فیزیکی، منظور ما از 
$g^{-1}$
همان 
$g^{\mu\nu} \partial_\mu\otimes\partial_\nu$
است.

ب)
برای هر بردار 
$X \in T_pM$
همواره 
\[
g^{-1} ( \alpha , X^{\flat}) = \alpha(X) = g(\alpha^\sharp,X)
\]
که در این عبارات، یک‌ریختی‌های موسیقایی فضای‌مماس
\LTRfootnote{Tangent space musical isomorphisms}
\footnote{یکر‌یختی فضاهای برداری را با یکریختی‌گروهی اشتباه نکنید؛ اینجا منظورمان یک‌ریختی فضای برداری‌استو}
به شکل زیر تعریف شده‌اند:
\begin{equation*}
	\begin{aligned}
		\flat &: T_pM \xrightarrow{\quad} T_p^*M \qquad X^\flat = g(X,.) \\
		\sharp&: T_p^*M \xrightarrow{\quad} T_pM \qquad \alpha^\sharp = g^{-1}(\alpha,.) 
	\end{aligned}
\end{equation*}
\end{parind}

\newpage
\textbf{سوال دوم: درباب هموستار و مشتق هموردا}

\vspace{0.7em}
 متریک
 $g$
 را متریک "هرمیتی" روی خمینه‌ی تقریبا مختلط
 \LTRfootnote{Almost complex manifold}
 $(M,J)$
  در‌نظر بگیرید.
  
  از این اصطلاحات نترسید؛ یک ساختار تقریبا مختلط
  \LTRfootnote{Almost complex structure}
  روی یک خمینه، نگاشتی است از هر فضای مماس به خودش به شکل زیر:
  \[
  \forall p\in M: \quad J(p): T_pM \xrightarrow{\quad} T_pM \;\; : \;\; J^2=-I_{n\times n}.
  \]
  و متریک $g$ هرمیتی است اگر 
  \[
  g(JX,JY) = g(X,Y), \qquad \forall X,Y \in \mathfrak{X}(M).
  \]
  
  تانسور 
  $(0,2)$
  (یعنی تانسوری که دو ورودیِ برداری می‌گیرد) $F$ را به شکل زیر تعریف می‌کنیم:
  \[
  F(X,Y) = g(X,JY)
  \]
  حالا ثابت کنید که:
  \begin{parind}
  	الف) $F$ پادمتقارن است.
  	
  	ب) $F$ به‌اصطلاح 
  	\lr{$J$-invariant}
  	 است؛ یعنی که 
  	 $F(JX,JY) = F(X,Y)$
  	
  	برای ادامه‌ی سوال، فرض کنید که هموستار سازگار با متریکِ
  	$\nabla$
  	مفروض هست.
  	
  	ج) نشان دهید
  	\[
  	(\nabla_X F)(Y,Z) = g(Y,(\nabla_XJ)Z).
  	\]
  	
  	د) نشان‌دهید
  	\[
  	g((\nabla_XJ)Y,Z)+ g(Y,(\nabla_XJ)Z) = 0.
  	\]
  \end{parind}


\newpage
\textbf{سوال سوم: محاسباتی در باب هموستار و ژئودزیک‌ها}

\vspace{0.7em}
فضای 
$\mathbb{R}^3$
 با متریک
 $ds^2 = (1+x^2)dx^2 + dy^2 + e^z dz^2$
 را به عنوان خمینه‌ی ریمانی معرفی می‌کنیم.
 
\begin{parind}
	الف) تمامی هموستار‌های لوی-چیویتا را پیدا کنید.
	
	ب) معادله‌ی ژئودزیک را حل کنید.
	
	ج) خم
	$\gamma(t) = (x=t,y=t,z=t)$
	 را درنظر داشته‌باشید. انتقالِ موازی هر بردار 
	 $(a,b,c)$ در مبدا، در راستای این خم را پیدا کنید.
	 
	 د) آیا خم $\gamma$ خود ژئودزیک است؟
	 
	 ه)دو میدان‌برداری موازی 
	 $X(t),Y(t)$
	 روی 
	 $\gamma$
	 پیداکنید که 
	 $g(X(t),Y(t))$
	 ثابت باشد.
\end{parind}

\newpage
\textbf{سوال چهارم: ساده و آموزشی درباره‌ی ژئودزیک}

\vspace{0.7em}
صفحه‌ی حقیقی با متریک اقلیدسی را به عنوان خمینه‌ی ریمانی معرفی می‌کنیم. آیا خم
$\gamma(t) = (t^3,t^3)$
یک ژئودزیک است؟

از این تمرین چه درسی می‌توان گرفت؟


\newpage
\textbf{سوال پنجم: درباب ایزومتری‌ها}

\vspace{0.7em}
نیم‌صفحه بالایی اعداد مختلط 
(
$\mathbb{H}$
)
 به همراه متریک پوانکاره
 $ds^2_{\text{\tiny Poincare}} = \frac{dz\wedge d\bar{z}}{\text{Im}^2(z)}$
 یک خمینه‌ی ریمانی است. گروه 
 $\text{SL}(2,\mathbb{R})$
 به شکل زیر روی نقاط این خمینه‌ی مختلط
 \footnote{خمینه‌های مختلط با حفظِ سمت، ریمانی هم هستند. بیشتر کارهایی که در خمینه‌های ریمانی انجام می‌دهیم تفاوت خاص و معناداری با خمینه‌های مختلط ندارد.}
  اثر می‌کند.
 \[
 A = \begin{pmatrix}
 	a & b \\ c& d
 \end{pmatrix} \in \text{SL}(2,\mathbb{R}), \qquad z \longmapsto \frac{az+b}{cz+d}
 \]
 نشان‌دهید که 
  $\text{SL}(2,\mathbb{R})$
  گروه ایزومتری‌های خمینه‌ی 
  $(\mathbb{H},ds^2_{\text{\tiny Poincare}})$
  است.

\newpage
\textbf{سوال ششم: فضاهای انحنا ثابت}

\vspace{0.7em}
خمینه‌ی 
$(M,g_M)$
را خمینه‌ای ریمانی با انحنای‌ثابت (
\lr{Maximally symmetric space}
)
بگیرید. روی خمینه‌ی حاصل‌ضربی
$M\times M$
متریک 
$\tilde{g}$
را به شکل زیر تعریف می‌کنیم.
\[
\tilde{g}((X_1,Y_1),(X_2,Y_2)) = g_M(X_1,X_2) + g_M(Y_1,Y_2), \quad X_1,X_2,Y_1,Y_2 \in \mathfrak{X}(M)
\]
یعنی متریک به شکل زیر است:
\[
\tilde{g}_{AB} = g_M \oplus g_M = \begin{pmatrix}
	g_M & 0 \\ 0 & g_M
\end{pmatrix}
\]
آیا خمینه‌ی 
$M\times M$
یک فضای انحنا ثابت است؟


\newpage
\textbf{سوال هفتم:  درباب متریک روی خمینه‌های لی}

\vspace{0.7em}
گروهِ لیِ هایزنبرگ را به خاطر آورید.
\[
H = \Biggl\{
\begin{pmatrix}
	1 & x & y \\ 0 & 1 & z \\ 0 & 0 & 1
\end{pmatrix} \; \Big| \; (x,y,z) \in \mathbb{R}^3
\Biggr\}
\]
\begin{parind}
	الف) متریک چپ‌ناوردای 
	$g$ را روی گروهِ لی
	$H$
	حاصل کنید؛ برای این کار از پایه‌های دوگان به میدان‌های برداری چپ‌ناوردا استفاده کنید.
	
	ب) هموستار لوی-چیویتا را روی این خمینه پیدا کنید.
	
	ج) آیا
	$(H,g)$
	یک فضای انحنا ثابت است؟
\end{parind}

\newpage
\textbf{سوال هشتم: باغ‌وحش عملگر‌ها روی خمینه‌ی ‌ریمانی}

\vspace{0.7em}
یک‌ریختی‌های موسیقایی را به خاطر آورید.
\begin{equation*}
	\begin{aligned}
		\flat: T_pM \xrightarrow{\quad} T_p^*M , \qquad X& \xrightarrow{\qquad } X^\flat \\ 
		X^\flat (Y&)= g(X,Y)
	\end{aligned}
\end{equation*}

\begin{equation*}
	\begin{aligned}
		\sharp: T_p^*M   \xrightarrow{\quad} T_pM , \qquad \omega& \xrightarrow{\qquad } \omega^\sharp \\ 
		\omega^\sharp (\xi&)= g^{-1}(\omega,\xi)
	\end{aligned}
\end{equation*}
همچنین گرادیان تابع را به شکل 
$\text{grad} f = (df)^\sharp$
تعریف می‌کنیم. موارد زیر را پیدا کنید.
\begin{parind}
	الف) 
	$g(\text{grad }f , X) = X[f]$
	
	ب)
	$(\frac{\partial}{\partial x^i})^\flat$
	
	ج)
	$(dx^i)^\sharp$
	
	د)
	گرادیان تابع $f$ را در مختصات موضعی بنویسید.
	
	ه) نشان‌دهید که در فضای تخت سه‌بعدی، همان عبارت آشنای گرادیان بدست می‌آید.
	 
\end{parind}

\newpage
\textbf{سوال نهم:
\lr{Operator Gymnastics}
}

\vspace{0.7em}
فضای 
$\mathbb{R}^n$
 با متریک اقلیدسی یک خمینه‌ی ریمانی است. فرمِ حجم
 $\omega = dx^1 \wedge \dots \wedge dx^n$
 است.
 \begin{parind}
 	الف) نشان دهید برای هر 
 	$\Omega_k \in \Lambda^k(\mathbb{R}^n)$
 	، تنها یک 
 	$(n-k)$-فرم
 	$\star\Omega_k$
 	وجود دارد که:
 	\[
 	\star\Omega_k (X_1 ,\dots ,X_{n-k}) \omega = \Omega_k \wedge X_1^\flat \wedge \dots \wedge X_{n-k}^\flat
 	\]
 	
 	ب) عملگر 
 	\lr{Hodge star}
 	\[
 	\star: \Lambda^k (\mathbb{R}^n) \longmapsto \Lambda^{n-k}(\mathbb{R}^n)
 	\]
 	مطابق قسمت قبلی تعریف می‌شود. نشان‌دهید که:
 	
 	\hspace{8mm}
 	ب۱) 
 	$\star^2 = (-1)^{k(n-k)}$
 	
 	\hspace{8mm}
 	ب۲)
 	$\star^{-1} = (-1)^{k(n-k)}\star $
 	
 	\hspace{8mm}
 	ب۳)
 	$\Omega_k \wedge (\star\Theta_k) = \Theta_k \wedge  (\star\Omega_k)$
 	
 	
ج) عملگر 
\lr{Codifferential}
 (
 $\delta : \Lambda^k(\mathbb{R}^n) \xrightarrow{\quad} \Lambda^{k-1} (\mathbb{R}^{n})$
 )
 به‌شکل زیر معرفی می‌شود:
 \[
 \delta = (-1)^{n(k+1)+1} \star d \star
 \]
 
 نشان‌دهید که 
 $\delta^2=0$، این یعنی که عملگر‌فوق می‌تواند یک 
 \lr{Cohomological complex}
 برایمان بسازد.
 
 د) لاپلاسی هم عملگری است که 
 $\Delta: \Lambda^k(\mathbb{R}^n) \xrightarrow{\quad} \Lambda^{k} (\mathbb{R}^{n})$
  و تعریف می‌شود:
  \[
  \Delta = (d+\delta)^2 = d\delta + \delta d
  \]
  نشان‌دهید برای هرتابع همواری مثل $f$ داریم:
  \[
  \Delta f = -\sum_{i=1}^n \frac{\partial^2 f}{\partial(x^i)^2}
  \]
 \end{parind}






\newpage
\textbf{سوال دهم:
تعریف بردار کیلینگ
\footnote{
این اسم منسوب به آقای «ویلهلم کارل جوزف کیلینگ» است.
}
}

\vspace{0.7em}
میدان‌برداری هموارِ
$X \in \mathfrak{X}(M)$  بردار کیلینگ است، اگر و فقط اگر 
$\mathcal{L}_X g = 0$.

\begin{mdframed}
	با‌این تعریف شروع کنید:
	
	بردار $X$ یک بردار کیلینگ است اگر متریک $g$ تحت نگاشت 
	\lr{pullback}
	در راستای 
	\lr{Integral curve}
	این بردار، ثابت بماند.
\end{mdframed}

\newpage
\textbf{سوال یازدهم:
	میدان‌های برداری کیلینگ 
	$\mathbb{R}^3$
}

\vspace{0.7em}
نشان‌دهید که 
جبر‌لی حقیقیِ تولید شده با
\[
\left<\partial_x , \partial_y , \partial_z , x\partial_y - y\partial_x, -y\partial_z + z\partial_y , z\partial_x - x\partial_z \right>
\]
میدان‌های برداری کیلینگ فضای 
$\mathbb{R}^3$
هستند.



\newpage
\textbf{سوال دوازدهم:
	کمی درباره‌ی عملگر‌های الحاقی
}

\vspace{0.7em}
خمینه‌ی ریمانی 
$(M,g)$
را فشرده فرض کنید.

\begin{parind}
	الف) نشان‌دهید که عملگر 
	\lr{codifferential}
	که در سوال نهم معرفی شده بود، الحاقیِ 
	\LTRfootnote{Adjoint}
	عملگر $d$ (مشتق خارجی) است، تحت ضرب‌ِداخلی القا شده با انتگرال‌گیری.
	\[
	\left< d\alpha,\beta \right> = 
	\left< \alpha , \delta \beta \right> , \qquad \alpha,\beta \in \Lambda^r(M)
	\]
	
	ب) عملگر لاپلاسی 
	$\Delta = d\delta +\delta d$
	تحت این ضرب‌داخلی، خودالحاقی
	\LTRfootnote{Self-adjoint}
	 است.
	\[
	\left< \Delta\alpha,\beta \right> = 
	\left< \alpha , \Delta \beta \right>
	\]

\end{parind}
\begin{mdframed}
	ضرب‌داخلی فرم‌ها با کمک انتگرال‌گیری چنین تعریف می‌شود:
	\[
	\left< \alpha,\beta \right> = \int_M (\alpha \wedge \star \beta) , \qquad \alpha,\beta \in \Lambda^r(M)
	\]
	آیا می‌توانید بگویید شرط فشرده بودن خمینه‌ی $M$ کجا لازم است؟
\end{mdframed}








\end{document}