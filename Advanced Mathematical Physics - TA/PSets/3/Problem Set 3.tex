\documentclass{article}
\usepackage{tcolorbox,mdframed}
\usepackage{physics, amsmath, amsfonts, fixmath, geometry, tikz, pgf, multirow, hyperref, amsfonts,amssymb, mathtools, physics, xcolor, siunitx, subcaption,tcolorbox}
\usepackage{graphicx}
\usepackage{pifont}
\usepackage{xcolor}
\usepackage{enumitem}
\usepackage{fontspec}
\usepackage{xepersian}
\settextfont[]{XB Zar}
\parindent 0mm

\newcommand{\mycomment}[1]{
	
}
%\setlatintextfont{Linux Libertine}
\setlength{\textwidth}{17.2 cm}
\setlength{\evensidemargin}{1 cm}
\setlength{\oddsidemargin}{-0.7 cm}
\linespread{1.9}
\title{ سری سوم تمرینات درس ریاضی‌فیزیک پیشرفته - دکتر کریمی‌پور
	\\
	\vspace{-1em}
	{\Large
		خمینه‌ها (جهت‌پذیری و انتگرال‌گیری فرمها) - گروه‌های لی}
		\\
		
}

\author{موعد تحویل پاسخ‌ها:
	دوشنبه ۳۱ اردیبهشت سال ۱۴۰۳ - تا ساعت 23:59
	\\
	از طریق 
	\href{https://cw.sharif.edu/}{سامانه درس‌افزار}
	\\
	دانشکده فیزیک - دانشگاه صنعتی شریف
}\date{}

\newcommand{\pardev}[2]{
	\frac{\partial #1}{\partial #2}
}

\begin{document}
	%\romantoday
	\maketitle
	
	\def\endline{		{
			\vspace{-2.5em}
			\color{cyan}
			\begin{center} \linethickness{1mm}\line(1,0){500} \end{center}
	}}
	\def\thinendline{		{
			\vspace{-2.5em}
			\color{purple}
			\begin{center} \linethickness{0.5mm}\line(1,0){500} \end{center}
	}}
	\vspace{-2em}
	\endline
	
	\noindent
	\textbf{تمرینات کلاسی:}\\
	\textbf{۱) اتحادهای ضرب خارجی}
	
	\noindent
	(
	در این روابط
	$\alpha \in \Lambda^p (\mathcal{M})$
	و 
	$\beta \in \Lambda^q (\mathcal{M})$
	و دیفئومورفیسم 
	$f:\mathcal{N} \longmapsto \mathcal{M}$
	مفروض‌اند.)
	\begin{equation}
		\begin{aligned}
			\text{d}(\alpha \wedge \beta) &= \text{d}\alpha \wedge \beta + (-1)^p \alpha \wedge \text{d}\beta \\
			f_*(\alpha \wedge \beta) &= f_*(\alpha) \wedge f_*(\beta) \\
			\text{d}(f_*(\alpha))                     &= 	f_*(\text{d}\alpha)
		\end{aligned}
	\end{equation}
	
	\vspace{-1em}
	
	
	\thinendline
	\vspace{-1em}
	\textbf{۲) اتحاد‌های ضرب داخلی}
	\begin{equation}
		\begin{aligned}
			\mathcal{L}_{X}\omega &= \text{d}(\iota_X\omega)  + \iota_X\text{d}\omega \\ 
			\iota_{[X,Y]} &= [\mathcal{L}_X,\iota_Y]
		\end{aligned}
	\end{equation}
	
	\vspace{-1em}
	\endline
	
	\vspace{-1em}
	\textbf{سوال اول:}
	با مطالعه‌ی تبدیلات بی‌نهایت کوچک گروه‌های زیر، تعداد پارامترها و پایه‌های جبر لی را پیدا کنید.
	\begin{equation*}
		\begin{aligned}
			O(n),\; SO(n),\; U(n),\; SU(n),\;\text{GL}(n,\mathbb{R}),\; \text{GL}(n,\mathbb{C}) ,\; \text{SL}(n,\mathbb{R}),\; \text{SL}(n,\mathbb{C})
		\end{aligned}
	\end{equation*}
	
	
	\vspace{-1em}
	\endline
	
	\vspace{-1em}
	\textbf{سوال دوم:}
	
	 \lr{Homeomorphism} 
	 بین خمینه‌های زیر را ثابت کنید
	 
	\begin{equation*}
		\begin{aligned}
			\frac{\text{SO}(3)}{\text{SO}(2)} &\cong S^2 \\
			\vspace{-1em}
			\frac{\text{SO}(n+1)}{\text{SO}(n)} &\cong S^n\\
			\frac{U(n+1)}{U(n)} &\cong S^{2n+1}\\
			\frac{\text{SU}(n+1)}{\text{SU}(n)} &\cong S^{2n+1}\\
			\frac{O(n+1)}{O(1)\times O(n)}	&\cong \mathbb{R}\text{P}^n\\
		\end{aligned}
	\end{equation*}
	\vspace{-1em}
	\endline
	
	\vspace{-2em}
	\noindent
	\textbf{سوال سوم:}
اگر 
$0<k< \min(m,n)$
باشد، نشان دهید مجموعه ماتریس‌های 
$m\times n$
با رتبه‌ی \underline{حداقل} 
$k$
 یک زیرخمینه از 
$M(m\times n , \mathbb{R})$
است
\footnote{منظور از 
$M(m\times n , \mathbb{R})$
، مجموعه‌ی تمامی ماتریس‌های 
$m\times n$ 
با درایه‌های حقیقی است.
}
.  نشان دهید اگر قید "حداقل" با "مساوی" جایگزین شود، خمینه‌ی حاصل    زیرخمینه‌ی
$M(m\times n , \mathbb{R})$
نیست.
	
	\vspace{-1em}
	\endline
	
	\noindent
	\textbf{سوال چهارم:}
	 نشان دهید که موارد زیر گروه لی هستند. 
	  
	  \noindent
	\text{الف) هر فضای برداری حقیقی متناهی بعد با عمل جمع میان بردارها.}\\
	\noindent
	\text{ب) فضای   
		$T^n = \underbrace{S^1 \times \dots \times S^1}_{\text{$n$ بار}}$
		   . (راهنمایی: آیا حاصل ضرب گروه‌های لی، یک گروه لی است؟)}\\
	ج) 
		$\text{Aut}(V)$
		 که $V$ فضای برداری متناهی بعد روی
		 $\mathbb{R}$
		 یا
		 $\mathbb{C}$
		 است
		 		 \footnote{
		 	منظور از 
		 	$\text{Aut}(V)$
		 	، تمامی خودریختی‌های
		 	(\lr{Automorphism})
		 	فضای برداری $V$ است؛ یعنی نگاشت‌های دوسویی فضای برداری که مبدا و مقصد آن خود 
		 	$V$ است.
		 }
		 . (عمل ضرب گروه، ترکیب است.)\\
	د) 
		$
			K = \mathbb{R}^n \times GL(n,\mathbb{R}) , n>1
		$
		با ضرب نیمه‌مستقیم گروهی
		\footnote{مشابه این ضرب را در ساختن گروه پوانکاره از گروه لورنتز می‌بینیم.}
		؛ یعنی 
		$\forall A,A' \in  GL(n,\mathbb{R}), \quad x,x'\in \mathbb{R}^n$ 
		\[
			(x,A)(x',A')= (x+Ax',AA') 
		\]
	
	
	\vspace{-2em}
	\endline
	
	\newpage
	\vspace{-2em}
	\textbf{سوال پنجم:}
		$\psi: G \mapsto G$
		  یک دیفیومورفیسم از گروه لی $G$ به خودش است. طوری که 
		  \[
		  \forall a \in G, \quad \psi(a) = a^{-1}
		  \]
		  نشان دهید $\omega$ فرم چپ-‌ناورداست اگر و تنها اگر 
		  $\psi_*\omega$ فرم راست-‌ناوردا باشد.
	
	\vspace{-2em}
	\endline
	
	
	\noindent
	\textbf{سوال ششم: }
	
	\noindent
	فرم دیفرانسیل زیر را در فضای 
	$\mathbb{R}^3$
	 در نظر بگیرید:
	\begin{equation*}
		\omega = (z-x^2-xy) \text{d}x\wedge\text{d}y - \text{d}y\wedge\text{d}z - \text{d}z\wedge\text{d}x
	\end{equation*}
	هم‌چنین دیسک دوبعدی را به شکل زیر تعریف می‌کنیم.
	\begin{equation*}
		D = \{(x,y) \in \mathbb{R}^2 \;|\; x^2 + y^2 \leq 1\} 
	\end{equation*}
	و نگاشت شمول
	\LTRfootnote{Inclusion}
	،
	$i:D\longmapsto \mathbb{R}^3$
	مفروض است (این نگاشت شمول، دیسک دو بعدی را روی صفحه‌ی 
	$z=0$
	می‌نشاند.)
	
	\text{انتگرال زیر را محاسبه کنید.}
	\begin{equation*}
		\int_{D}^{} i_*\omega
	\end{equation*}
	
	\vspace{-2em}
	\endline
	
	\textbf{سوال هفتم:}
	۲-فرم زیر را درنظر بگیرید.
		\begin{equation*}
		\alpha = \frac{x\text{d}y\wedge\text{d}z - y\text{d}x\wedge\text{d}z + z\text{d}x\wedge\text{d}y}{(x^2 + y^2 + z^2)^\frac32} \in \Lambda^2(\mathbb{R}^3 - \{0\})
\end{equation*}

الف) نشان دهید $\alpha$ فرم بسته است
	\footnote{فرم 
	$\omega$، طبق تعریف، 
	بسته (\lr{closed}) است اگر 
	$\text{d}\omega = 0$.
	}
	.\\
ب) انتگرال زیر را محاسبه کنید و بگویید چطور این پاسخ نشان می‌دهد که $\alpha$ کامل
	\footnote{فرم 
		$\omega$، طبق تعریف، 
		کامل (\lr{exact}) است اگر فرم مرتبه پایین‌تری مثل 
		$\xi$
		موجود باشد که  
		$\omega = \text{d}\xi$.
	}
	 نیست
	 \footnote{انتگرال‌گیری روی کره‌ی واحد به مرکز مبدا مختصات صورت می‌گیرد.}
	 .
\begin{equation*}
	\int_{S^2}^{} \alpha
\end{equation*}

\vspace{-2em}
\endline

	\textbf{سوال هشتم:}
	قرار دهید
	\begin{equation*}
		\begin{aligned}
			 H = \Bigl\{
			\left( \begin{smallmatrix}
			 	1 &‌x &y \\ 0&1&z \\ 0&0&1
			 \end{smallmatrix} \right) \;\;\Big| \;\; x,y,z \in \mathbb{R}^3
			 \Bigr\}.
		\end{aligned}
	\end{equation*}
	
	\noindent
	الف) نشان دهید که $H$ ساختار یک خمینه‌ی هموار را دارد که با 
	$\mathbb{R}^3$
	دیفئومورفیک است.
	
	\noindent
	ب) نشان دهید که $H$‌ با عمل ضرب ماتریسی یک گروه لی است. (به این گروه، گروه هایزنبرگ می‌گوییم.)
	
	\noindent
	ج)نشان دهید که مجموعه‌ی 
	$ L = \bigl\{
	\frac{\partial}{\partial x} , \frac{\partial}{\partial y} , x\frac{\partial}{\partial y} + \frac{\partial}{\partial z}
	\bigr\}$
	پایه‌ای برای جبرلی 
	$\mathfrak{h}$
	(جبر لی گروه $H$) است.
	
	\vspace{-2em}
	\endline
\end{document}




