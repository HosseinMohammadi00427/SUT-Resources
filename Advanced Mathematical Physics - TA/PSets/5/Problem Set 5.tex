\documentclass{article}
\usepackage{tcolorbox,mdframed}
\usepackage{physics, amsmath, amsfonts, fixmath, geometry, tikz, pgf, multirow, hyperref, amsfonts,amssymb, mathtools, physics, xcolor, siunitx, subcaption,tcolorbox}
\usepackage{graphicx}
\usepackage{pifont}
\usepackage{xcolor}
\usepackage{enumitem}
\usepackage{fontspec}
\usepackage{xepersian}
\settextfont[]{XB Zar}
\parindent 0mm
\newenvironment{parind}{%
	\par%
	\medskip
	\leftskip=0mm\rightskip=7mm
	\noindent\ignorespaces}{%
	\par\medskip}
\newcommand{\mycomment}[1]{
	
}
%\setlatintextfont{Linux Libertine}
\setlength{\textwidth}{17.2 cm}
\setlength{\evensidemargin}{1 cm}
\setlength{\oddsidemargin}{-0.7 cm}
\linespread{1.9}
\title{ سری پنجم(آخر) تمرینات درس ریاضی‌فیزیک پیشرفته - دکتر کریمی‌پور
	\\
	\vspace{-1em}
	{\Large
		گروه‌های کوهمولوژی و کوهمولوژی
		\lr{De-Rham}
		}
		\\
		
}

\author{موعد تحویل پاسخ‌ها:
	یک‌هفته پس از مهلت آخرین سری تمرین‌ها - تا ساعت 23:59
	\\
	از طریق 
	\href{https://cw.sharif.edu/}{سامانه درس‌افزار}
	\\
	دانشکده فیزیک - دانشگاه صنعتی شریف
}\date{}

\newcommand{\pardev}[2]{
	\frac{\partial #1}{\partial #2}
}

\begin{document}
	%\romantoday
	\maketitle
	
	\def\endline{		{
			\vspace{-2.5em}
			\color{cyan}
			\begin{center} \linethickness{1mm}\line(1,0){500} \end{center}
	}}
	\def\thinendline{		{
			\vspace{-2.5em}
			\color{purple}
			\begin{center} \linethickness{0.5mm}\line(1,0){500} \end{center}
	}}
\textbf{
دقیقا به دو سوال از سوال‌های زیر پاسخ بدهید.
}
	\vspace{-1.5em}
	\endline

	\vspace{-3em}
	\noindent
	\textbf{سوال اول:}
		زیرمجموعه‌یِ بازِ
		$ U \subset \mathbb{R}^n$
		 ستاره‌ای-شکل
		\LTRfootnote{Star-shaped}
		است که صفر را در بر دارد. 
		\lr{$p$}-فرمِ بسته‌ی 
		$\omega = \sum \omega_I dx^I$
		روی $U$ تعریف شده است
		\footnote{$I$ مجموعه‌ای از $p$ اندیس است که به دلخواه مرتب شده است.}
		. 
		نشان‌دهید 
		$(p-1)$- فرمِ
		$\eta$ در رابطه
		$d\eta = \omega$
		صدق می‌کند.
		\begin{equation*}
			\begin{aligned}
				\sum_I \sum_{q=1}^{p} (-1)^{q-1} \Big(
				\int_0^1 t^{p-1} \omega_I(tx) dt\Big) x^{i_q} dx^{i_1} \wedge dx^{i_2}\wedge\dots\wedge \widehat{dx^{i_q}} \wedge\dots \wedge dx^{i_p}
			\end{aligned}
		\end{equation*}

	درحقیقت، با این‌کار قضیه پوانکاره را نشان می‌دهید؛ یعنی هر فرمِ بسته روی زیرفضایِ بازِ 
	$\mathbb{R}^n$
	(با گروه‌های همولوژی بدیهی)
	یک فرمِ کامل است. 
	\vspace{-3em}
	\endline
	
	\vspace{-3em}
	\textbf{سوال دوم:}
	خمینه‌های هموار
	$M_1$و $M_2$ را خمینه‌هایی همبند، فشرده و جهت‌پذیر با بعدِ $n\geq 2$ درنظر بگیرید.
	نشان‌دهید که 
	\[H^k_{\text{dR}}(M_1\# M_2) = H^k_{\text{dR}}(M_1) \oplus H^k_{\text{dR}}(M_2).\]
	(
	برای
	$0<k<n$.
	)
	
	\vspace{-1em}
	\endline
	
	\vspace{-1em}
	\noindent
	\textbf{سوال سوم:}
	$M$ را خمینه‌ای هموار بگیرید و
	فرم‌های 
	$\omega \in \Lambda^p(M)$
	و
	$\eta \in \Lambda^q(M)$
	 مفروض هستند.
	 اول نشان‌دهید که کلاسِ کوهمولوژیِ فرم 
	 $\omega \wedge \eta$
	  تنها به کلاسِ کوهمولوژیِ 
	  $\eta$ و‌$\omega$
	   بستگی دارد.  بنابراین نگاشتِ 
	    $\smile : H^p_{\text{dR}}(M_1) \times H^q_{\text{dR}}(M_1) \xrightarrow{\quad} H^{p+q}_{\text{dR}}(M_1)$
	    با تعریفِ
	    \newpage
	    \[
	    [\omega] \smile [\eta] = [\omega \wedge \eta]
	    \]
	    خوش‌تعریف است. به این ضرب، 
	    \lr{Cup product}
	    گفته می‌شود.
	    
	\vspace{-1em}
	\endline
	
	\vspace{-1em}
	\textbf{سوال چهارم: بدست آوردن گروه‌های کوهمولوژی}
	\begin{parind}
		الف)
		تمامی گروه‌های کوهمولوژی فضای 
		$\mathbb{R}^2 - \{p_1,p_2\}$
		را حساب کنید. (یعنی صفحه‌ی حقیقی که دو نقطه‌ی دلخواه از آن حذف شده اند.)
		
		ب)
		گروه‌های کوهمولوژی ناحیه‌ی 
		\[M= \biggl\{
		(x,y) \in \mathbb{R}^2 \; | \; 1 < \sqrt{x^2+y^2} <2
		\biggr\}
		\]
		را بدست بیاورید.( از این نکته استفاده کنید که اگر بین دو خمینه‌ی هموار یک هموتوپی باشد، آن‌گاه تحت \lr{pullback} کردن‌ کلاس‌های کوهمولوژی با هموتوپی بین این دو خمینه، گروه‌های کوهمولوژی دو خمینه یک‌ریخت هستند. سپس، یک هموتوپی بین ناحیه‌ی حلقوی بالا و صفحه با یک سوراخ معرفی کنید.)
		
		
		\vspace{-1em}
		\endline
	\end{parind}
	\vspace{-1em}
	\textbf{سوال پنجم:}
	با فضای 
	$\mathbb{C}P^n$ آشنا هستید و مختصات‌های روی آن را دیده‌اید. اما به عنوان یادآوری نگاهی به آن بیندازیم:
	\begin{mdframed}
		بازِ
		$U_\alpha$ را  که با مختصات
		 $z_i\neq 0$
		 برای
		 $i =  \alpha$
		 تعریف می‌شود، به یاد آورید. مختصات‌های بهنجارِ
		 $u^j = z^j/z^\alpha$ 
		 و 
		 $v^j = z^j/z^\beta$
		 روی دو بازِ
		 $(U_\alpha,U_\beta)$
		 تعریف کنید.
		 
		 رویِ 
		 $U_\alpha$
		 و
		 $U_\beta$
		 ، دو-فرم‌های زیر را تعریف می‌کنیم:
		 \begin{equation*}
		 	\begin{aligned}
		 		\omega_\alpha &= -i \Big(
		 		\frac{\sum_{j} du^j \wedge d\bar{u}^j}{\varphi} - \frac{\sum_{j,k}u^j\bar{u}^k du^k \wedge d\bar{u}^j}{\varphi^2}
		 		\Big) \\
		 		\omega_\beta &= -i \Big(
		 		\frac{\sum_{j} dv^j \wedge d\bar{v}^j}{\psi} - \frac{\sum_{j,k}v^j\bar{v}^k dv^k \wedge d\bar{v}^j}{\psi^2}
		 		\Big), \\ 
		 	\end{aligned}
		 \end{equation*}
		 که در آن‌ها
		 $\varphi = \sum_{j=0}^{n} u^j \bar{u}^j$
		   و
		   $\psi = \sum_{j=0}^{n} v^j \bar{v}^j$.
	\end{mdframed}
	\begin{parind}
		الف) نشان‌دهید که تحدید این دو-فرم‌ها به ناحیه‌ی اشتراک، یکسان است.
		\[
		\omega_\alpha \Big|_{U_\alpha \cap U_\beta} = \omega_\beta \Big|_{U_\alpha \cap U_\beta}
		\]
		
		\newpage
		ب) یک 2-فرمِ یکتا رویِ
		$\mathbb{C}P^n$
		هست، که آن را 
		$\omega$
		 می‌نامیم. با این خاصیت که 
		 $\omega\Big|_{U_\alpha} = \omega_\alpha$.
		 
		 ج) $d\omega=0$
		 
		 د) 
		 $\wedge_n \omega$
		 یک فرمِ حجم است.
		 
		 ه) $\omega$ یک فرمِ کامل نیست. 
	\end{parind}
		\vspace{-1em}
	\endline
\end{document}




