\documentclass{article}
\usepackage{tcolorbox,mdframed}
\usepackage{physics, amsmath, amsfonts, fixmath, geometry, tikz, pgf, multirow, hyperref, amsfonts,amssymb, mathtools, physics, xcolor, siunitx, subcaption,tcolorbox}
\usepackage{graphicx}
\usepackage{pifont}
\usepackage{xcolor}
\usepackage{enumitem}
\usepackage{fontspec}
\usepackage{xepersian}
\settextfont[]{XB Zar}
\setdigitfont{XB Zar}

%\setlatintextfont{Linux Libertine}
\setlength{\textwidth}{17.2 cm}
\setlength{\evensidemargin}{1 cm}
\setlength{\oddsidemargin}{-0.7 cm}
\linespread{1.9}
\title{ سری دوم تمرینات درس ریاضی‌فیزیک پیشرفته - دکتر کریمی‌پور
\\
\vspace{-1em}
توپولوژی، همولوژی و هموتوپی
}

\author{موعد تحویل پاسخ‌ها:
 دوشنبه ۱۷ اردیبهشت سال ۱۴۰۳ - تا ساعت 59:23
 \\
 از طریق 
 \href{https://cw.sharif.edu/}{سامانه درس‌افزار}
\\
دانشکده فیزیک - دانشگاه صنعتی شریف
}\date{}

\newcommand{\pardev}[2]{
	\frac{\partial #1}{\partial #2}
}

\begin{document}
	%\romantoday
	\maketitle
	
	\def\endline{		{
			\vspace{-2.5em}
			\color{cyan}
			\begin{center} \linethickness{1mm}\line(1,0){500} \end{center}
	}}
\def\thinendline{		{
		\vspace{-2.5em}
		\color{purple}
		\begin{center} \linethickness{0.5mm}\line(1,0){500} \end{center}
}}
\vspace{-2em}
		\endline
		
		\noindent
		\textbf{سوال اول:}
		نشان دهید
	 زیرمجموعه $A$ از فضای متریک کامل $(M,d)$ فشرده است، اگر و تنها اگر بسته و تماما کراندار
	 \LTRfootnote{Toatally bounded}
	  باشد.
	 
	 \begin{mdframed}
	 	\textbf{تعریف تماما کرانداری:}
	 	زیرفضای $A$ از فضای متریک $(M,d)$ تماما کراندار است اگر برای هر 
	 	$\varepsilon>0$
	  مجموعه‌ای متناهی از گوی‌های باز با شعاع $\varepsilon$ باشد که:
	  \begin{enumerate}
	  	\itemsep-0.5em 
	  	\item مرکز گوی‌ها در $A$ باشد.
	  	\item اجتماع تمامی این گوی‌های باز شامل کل $A$ شود. 
	  \end{enumerate}
	 \end{mdframed}
		
		\vspace{-2em}
				\endline
		
		\noindent
		\textbf{سوال  دوم:}
		دو فضای متریک 
		$(M,d_M)$ 
		و
		$(N,d_N)$
		مفروض هستند.
		اگر نگاشت 
		$f: M \longmapsto N$
		پیوسته باشد و $A$ زیرمجموعه‌ای فشرده از $M$ باشد؛ نشان دهید که تصویر $A$، یعنی $f(A)$ زیرمجموعه‌ای فشرده از $N$ است. نتیجه بگیرید که نگاشت
		\lr{Homeomorphism}
		بین دوفضای متریک، وضعیت فشردگی را حفظ می‌کند.
		
		
				\endline
		
		\noindent
		\textbf{سوال سوم:}
		ساده‌ترین نگاشت بین فضاهای متریک یک 
		\textit{ایزومتری}
		است؛ یعنی یک تابع دوسوییِ
		$f: M \longmapsto N$ 
		که فاصله‌نگهدار است؛ یعنی برای هر دو عضو $p,q\in M$ داریم:
		\[
		d_N(f(p),f(q)) = d_M(p,q)
		\]
		همچنین به فضاهایی که بین‌شان نگاشت ایزومتری موجود باشد، فضاهای \textit{ایزومتریک} می‌گوییم.
		
		\noindent
		الف) نشان دهید که نگاشت ایزومتری، پیوسته است.
		
		\noindent
		ب) استدلال کنید که هر نگاشت ایزومتری یک 
		\lr{Homeomorphism}
		است.
		
		\noindent
		ج) نشان دهید که بازه‌ی حقیقی
		$[0,1]$
		 با بازه‌ی 
		 $[0,2]$
		 ایزومتریک نیست.
		 
		\endline
		
				\noindent
		\textbf{سوال چهارم:}
		اگر $M$ و $N$ دو فضای متریک ناتهی باشند. نشان دهید که اگر هر دو همبند
		\lr{Connected}
		باشند، فضای متریک 
		$M\times N$ هم همبند است.
		در مورد عکس این گزاره استدلال یا مثال نقضی بیاورید.
		
			\endline
			
							\noindent
			\textbf{سوال پنجم:}
			به کمک تعاریف، گزاره‌های زیر را برای یک زیرفضای $S$ از فضای متریک $M$ نشان دهید.
			
			\noindent
			الف)
			$S^{\circ} = S\backslash \partial S$
			
			\noindent
			ب) 
			$(S^{\circ})^\circ = S^{\circ}$
			
			\noindent
			ج)
			$\partial S = \partial \bar{S}$
			
			 \noindent
			 د)
			 $\partial\partial S \subset \partial S$
			
			
			
			
			\endline
			
			
							\noindent
			\textbf{سوال ششم:}
			فرض کنید
			$\pi: \tilde{X} \longmapsto X$
			یک نگاشت پوششی باشد که
			برای هر $x\in X$، 
			$\pi^{-1}(x)$
			ناتهی و متناهی باشد. نشان دهید که فضای توپولوژیک $\tilde{X}$ هاسدورف 
			\LTRfootnote{Hausdorff}
			است اگر و فقط اگر فضای توپولوژیک
			$X$
			هاسدورف باشد.
			
			\endline
			
			
							\noindent
			\textbf{سوال هفتم:}
			گروه‌های همولوژی نوار موبیوسی که یک نقطه از آن حذف شده را پیدا کنید.
			
			
			\endline
			
			\newpage
							\noindent
			\textbf{سوال هشتم:}
			$\pi_1(\Sigma_{g,n})$، یعنی 
			گروه هموتوپی اول یک رویه‌ی ریمانی با جینس $g$ که $n$ نقطه‌ی آن حذف شده‌اند، پیدا کنید. آیا می‌توانید با آبلی کردن گروه بدست‌آمده، به گروه‌ همولوژی اول برسید؟
			
			\endline
			
			
							\noindent
			\textbf{سوال نهم:}
			نشان دهید که بین فضاهای زیر،  
			\lr{Deformation Retraction}
			وجود ندارد.
			
			\noindent
			الف) 
			$\mathbb{R}^2$
			و
			$S^1$.
			
			\noindent
			ب) 
			$S_1 \times D_1$
			و
			$S_1\times S_1$. 
			(یعنی از یک چنبره‌ی توپر به مرزش)

			\noindent
			ج) نوار موبیوس و مرزش.
			
			\endline
			
							\noindent
			\textbf{سوال دهم:}
			مجموعه‌ی 
			$X$ را متشکل از 
			$n$ خط مبداگذر در 
			$\mathbb{R}^3$
			بگیرید. گروه
			$\pi_1(\mathbb{R}^3-X)$
			را بدست آورید.
			
			\endline

\end{document}



