\documentclass{article}
\usepackage[a4paper,bindingoffset=0.2in,
            left=1in,right=1in,top=1in,bottom=1in,
            footskip=.25in]{geometry}
\usepackage{amsmath }
\begin{document}
\centering Student Name : Hossein Mohammadi \\
\centering Student No : 96101035\\
\centering Computational Physics Course\\
\centering TA : Ali Farnoodi  
\\[2cm]
\raggedright
In order to drive the $\sigma^{2}$ for this problem, first we know that:
\begin{equation}
<x(t)> = Nl(p-q)
\end{equation}
As it calculated in Lecture Notes of Random Walk . Next we should Find $<x^{2}(t)>$:

\begin{multline*}
\begin{split}
 <x^{2} (t) > = \left<l\left(\sum\limits_{i=1}^N a_{i}\right) l\left(\sum\limits_{j=1}^N a_{j}\right)\right>  
&=l^{2} \left<\sum\limits_{i=1}^N\sum\limits_{j\neq i}^N a_{i}a_{j}  + \sum\limits_{i=1}^N a_{i}^{2}\right> = 
 l^{2} \left<\sum\limits_{i=1}^N\sum\limits_{j\neq i}^N a_{i}a_{j} \right>+ l^{2} \left< \sum\limits_{i=1}^N a_{i}^{2}\right> \\
l^{2}\sum\limits_{i=1}^N\sum\limits_{j\neq i}^N <a_{i}a_{j} >  + l^{2}\sum\limits_{i=1}^N <a_{i}^{2}>
\end{split}
\end{multline*}

Now, Since $a_{i}$ is a random variable that has values 1 and -1 with specific probablities p and q, $a_{i}^2$ is always equal to 1 and since $a_{i}$ is independent of $a_{j}$ for $i\neq j$ it's obvious
that we can write $<a_{i}a_{j}>$ = $<a_{i}><a_{j}>$. By using this we can write:

\begin{multline*}
\begin{split}
l^{2} \left( \sum\limits_{i=1}^N\sum\limits_{j\neq i}^N <a_{i}a_{j} >  + \sum\limits_{i=1}^N <a_{i}^{2}>\right) = 
l^{2} \left( \sum\limits_{i=1}^N\sum\limits_{j\neq i}^N <a_{i}><a_{j} >  + \sum\limits_{i=1}^N 1 \right)
\end{split}
\end{multline*}
For each i, j is summed over N-1 times (j=i is neglected) and recall that $<a_{i}> = p-q $ so we can rewrite the whole first term as:
 

\begin{multline*}
\begin{split}
l^{2} \left(\sum\limits_{i=1}^N\sum\limits_{j\neq i}^N <a_{i}><a_{j} >  + \sum\limits_{i=1}^N 1\right) =
l^{2}(N(N-1)(p-q)^{2} +N)
\end{split}
\end{multline*}
So by : $\sigma^{2} = <x^2(t)> - <x(t)>^2$ , We write:


\begin{multline*}
\begin{split}
\sigma^{2} =  l^{2} ( N^2 (p-q)^{2} -N(p-q)^2 + N) - N^2 l^2 (p-q)^2 =Nl^2(1-(p-q)^2) = Nl^2(1+p-q)(1-p+q)
\end{split}
\end{multline*}
\\[0.5cm]
But $p+q = 1$:

\begin{multline*}
\begin{split}
\sigma^{2} =  Nl^2(1+p-q)(1-p+q) = Nl^2(2p)(2q) =4 Nl^2 pq
\end{split}
\end{multline*}
Which is like the equation that mentioned in lecture note but with a little deiffrence: I replaced $\frac{t}{\tau }$ with N which is the number of steps. So I prove that formula for the standard variance 
in Random Walk.


\end{document}



















