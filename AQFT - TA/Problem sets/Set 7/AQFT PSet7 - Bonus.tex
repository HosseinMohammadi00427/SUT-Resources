\documentclass[11pt]{article}
\usepackage{physics,slashed}




% NOTE: Add in the relevant information to the commands below; or, if you'll be using the same information frequently, add these commands at the top of paolo-pset.tex file. 
\newcommand{\name}{TA: Hossein Mohammadi}
\newcommand{\email}{hossein.mohammadi.00427@gmail.com}
\newcommand{\classnum}{Advanced Quantum Field Theory}
\newcommand{\subject}{Subject: Renormalization Group Equation of QED and $\phi^4$ theory}
\newcommand{\instructors}{Dr. Amin Faraji}
\newcommand{\assignment}{Bonus PSet}
\newcommand{\semester}{- Fall 1402}
\newcommand{\duedate}{dd/mm/yyyy}

\input{paolo-pset.tex}

% NOTE: To compile a version of this pset without problems, solutions, or reflections, uncomment the relevant line below.

%\excludeversion{problem}
%\excludeversion{solution}
%\excludeversion{reflection}

\begin{document}	
	
	% Use the \psetheader command at the beginning of a pset. 
	\psetheader
	
	\section*{Problem 1: QED beta-function and Callen-Symanzik Equation}
	
	\begin{problem}
		We have promised that renormalization has more significant physical consequences than just logarithmic dependence of amplitudes in different energy scales. Renormalization Group Equation (R.G.E.) is another physical result of renormalization.
		
		Renormalization group\footnote{This "group" has nothing to do with "group theory"!} refers to invariance of observables under the change in the way the amplitudes are calculated. As an instance, in Dimensional Regularization (D.R.),several amplitudes were $\mu-$ dependent, so R.G.E. should eliminate this dependence in the observables.
		
	\end{problem}
	\begin{enumerate}
		\item
		\begin{problem}{\points{-}}
			\textbf{The Running of the Coupling Constant:}
			
			Recall that where the $\mu-$factor dependence came from in the renormalized perturbation theory of QED, we added the arbitrary scale $\mu$ to render the coupling constant $e$ dimensionless. So it appeared in the following terms of the Lagrangian:
			\[
			\mathcal{L}_{int}^{R.P.T} = -\mu^{\frac{4-d}{2}} e_R Z_e Z_2\sqrt{Z_3} \bar{\psi}^R\slashed{A}^R \psi^R.
			\]
			And also remember from the former problem set that the counterterms in $e^2_R$ order were:
			\[
			\begin{pmatrix}
				\delta_2 \\
				\delta_3 \\
				\delta_m \\
				\delta_e
			\end{pmatrix}
			 = 
			 \frac{e^2_R}{16\pi^2}
			 \begin{pmatrix}
			  \vspace{0.75 em}	-\frac{2}{\varepsilon} \\ \vspace{0.75 em}
			 	-\frac{8}{3\varepsilon} \\ \vspace{0.75 em}
			 	-\frac{6}{\varepsilon} \\ \vspace{0.75 em}
			 	\frac{4}{3\varepsilon}
			 \end{pmatrix}
			\]
			Use the $\mu-$independence of the bare charge $e$\footnote{$\mu\frac{d}{d\mu}e_0 = 0$}, then replace $e_0$ by $e_RZ_e$ to find the coupling of the renormalized charge $e_R$:
			\begin{itemize}
				\item At leading order.
				
				 (\textbf{Hint:} Reach to the following equation:
				$
				\mu\frac{d}{d\mu}e_R = -\frac{\varepsilon}{2}e_R 
				$
				)
				\item Next to leading order.
				
				(\textbf{Hint:} Reach to the following equation:
				$
				\mu\frac{d}{d\mu}e_R = -\frac{\varepsilon}{2}e_R  + \frac{e^3_R}{12\pi^2}
				$
				)
			\end{itemize}
			\item 
			\begin{problem}{\points{-}}
				\textbf{The Running of the electron mass:}
				
			Find the R.G.E. for the electron mass by the same method.
			
			(\textbf{Hint:} Define $\gamma_m \equiv \frac{\mu}{m_R} \frac{dm_R}{d\mu}$, which is called "Anomalous dimension" conventionally.)
		\end{problem}
			
			
			\item 
			\begin{problem}{\points{-}}
				\textbf{R.G.E. for n-point Functions:}
				
				Consider the following Green-functions, comprising of $n$ photons and $m$ fermions in the bare theory:
				\[
				G_{n,m}^0 = \bra{\Omega} T\{
				A_{\mu_1}^0 \dots A_{\mu_n}^0
				\; 
				\psi_1^0 \dots \psi_n^0
				\}
				\ket{\Omega}
				\]
				\begin{enumerate}
					\item Find the relation between bare Green-function and the renormalized one, $G_{n,m}^R$.
					\item Define $\gamma_i \equiv \frac{\mu}{Z_i} \frac{dZ_i}{d\mu}$ for $i=2,3$, then use R.G.E. to find the Callen-Symanzik equation:
					\[
					\big(
					\mu \frac{\partial}{\partial \mu} + \frac{n}{2} \gamma_3 + \frac{m}{2}\gamma_2 + \beta \frac{\partial}{\partial e_R} + \gamma_m m_R \frac{\partial}{\partial m_R} 
					\big) G_{n,m}^R =0
					\]
				\end{enumerate}
			\end{problem}
		\end{problem}
	\end{enumerate}
	
	\newpage

	\section*{Problem 2: $\phi^4$-theory R.G.E. }
 


\begin{problem}
	We have worked out the counterterms associated with $\phi^4$ theory at one-loop level (or $\lambda_R^2$ order.)
	
	For this problem, work out the R.G.E. for $m_R$ and $\lambda_R$ in this theory. Then find the running of these parameters with $\mu$ scale by solving the R.G.E. differential equation.
	\end{problem}












\end{document}